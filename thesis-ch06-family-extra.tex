% This is really cool extra stuff that unfortunately I just don't have time for!

\subsection{Grasped Objects}
\label{subsec:family:grasped-objects}

One instance specific to manipulation problems is the handling of
grasped objects.
For example, 
consider a manipulator which grasps a geometric object.
This affects the set of collision-free configurations
across a large section of $\mathcal{C}$
relative to the old set of valid configurations $S_{\ms{old}}$.
However,
the resulting $\mathcal{C}$-subset $S_{new}$
can be represented simply as
$S_{\ms{new}} = S_{\ms{old}} \cap G$,
with $G$ the set of robot configurations in which
the \emph{grasped object} (only)
is deemed free of collision with the robot and environment.
This structure is discussed in the context of the
\emph{conditional reachability graph},
part of the \textsc{FFRob} heuristic framework
\citep{garrett2014ffrob}.

For example,
consider the manipulation problem in
Figure~\ref{fig:family:testherb-problem}.
The robot must find a path which moves its arm to grasp the cup.
After the cup is grasped,
the robot can reuse any edge in the existing roadmap
by simply checking the grasped cup
against the remainder of the environment.
This structure,
together with the approach to dynamic environments,
are included together in the experimental results
(Section~\ref{subsec:family:herb-experiment}).

\section{Application: Caching Invariant Geometry}
\label{subsec:family:cached-self-valid}

\paragraph{Cached Self-Collision-Checked Roadmaps}
Self-collision checking is a potentially expensive component to
articulated motion planning;
in contrast to environment checking,
it is fundamentally quadratic in the number of moving links.
Further, pairs of links to be checked
tend to be relatively close to each other,
reducing the effeciveness of broad-phase approaches.

Leven and Hutchinson \citep{leven2000changing}
introduced the concept of a pre-cached roadmap consisting of
configurations and paths already known to be valid w.r.t.
self-collision.
As a type of invariant in $\mathcal{C}$,
this can be seen as a particular instance of family motion planning.
See Fig.~\ref{fig:family:self-collision-example}.

\begin{marginfigure}
   \centering
   \includegraphics{build/self-collision}
   \caption{A roadmap is pre-computed in $R$,
      the subset of $\mathcal{C}$ consisting of configurations free
      of robot self-collision.
      Online, the planner must find a path that's also within $E$,
      the subset free of environment collision.
      When solving this query in $S = R \cap E$,
      the Family PRM automatically prefers potential paths with
      pre-computed edges (e.g. shown in grey)
      due to lower planning costs over alternatives with lower
      execution costs.}
   \label{fig:family:self-collision-example}
\end{marginfigure}

\begin{figure*}
   \centering   
   \includegraphics{build/family-sq-caching/pvxs}
   \caption[LEMUR single-query results on several HERB problems,
      all using baked calls.
      Traces are LEMUR,
      the family motion planner (no caching),
      and the family motion planner with self-checked caching.
   ]{LEMUR single-query results on several HERB problems,
      all using baked calls.
      Traces are LEMUR
      (\protect\tikz{\protect\draw[thick,black] (0,0) -- (0.15,0.15);}),
      the family motion planner (no caching)
      (\protect\tikz{\protect\draw[thick,blue] (0,0) -- (0.15,0.15);}),
      and the family motion planner with self-checked caching
      (\protect\tikz{\protect\draw[thick,green] (0,0) -- (0.15,0.15);}).}
\end{figure*}





\section{Application: Exploiting Conservative Geometric Approximations}
\label{subsec:family:broad-phase}

The family formulation also enables motion planners to
reason directly about different robot or environment models.
For example, consider two geometric robot models,
one with high quality (e.g. from a CAD program),
and one hand-tuned ``padded'' model consisting of 
a small number of simple conservative bounding volumes.
The $\mathcal{C}$-subsets derived from these models
are related by $R_{\ms{padded}} \subseteq R_{\ms{CAD}}$.
Collision checkers currently use a similar approach internally
to speed up collision checks (see Fig.~\ref{fig:family:broad-phase}.
and Fig~\ref{fig:family:broad-phase-2d}).

See Figure~\ref{fig:family:broad-phase-2d}
for a simple example with integrated broad-phase collision checking
as described in Section~\ref{subsec:family:broad-phase}.

\begin{figure*}[b]
   \centering
   
   \subfloat[
      A single-set planner testing simply for membership in
      $\mathcal{C}_{\mbox{\scriptsize free}}$
      treats a collision validity checker as a
      ``black box.''
      Internally,
      modern checkers first employ an inexpensive broad-phase check
      using a low-dimensional conservative representation
      to quickly identify non-colliding bodies before
      resorting to an expensive narrow-phase check.
   ]{%
      \includegraphics{build/broadphase-single}%
   }%
   \quad%
   \subfloat[
      A family motion planner can explicitly reason about the
      conservative nature of the broad-phase check.
      This allows it to defer some narrow phase checks
      (often indefinitely)
      and instead prefer paths that require fewer expensive checks.
   ]{%
      \includegraphics{build/broadphase-multi}%
   }
   
   \caption[][0.0in]{Collision validity checking is a commonly used
     indicator function.
     The family formulation allows an intelligent planner to
     reach inside the checker's ``black box'' and reduce the number
     of costly narrow-phase checks.
     Resulting paths tend to be cheaper to compute and
     stay further from obstacles.}
   \label{fig:family:broad-phase}
\end{figure*}

% this is the broad phase bean figure
\begin{figure}
   \centering

   % left side
   \subfloat[Paths with lambda=0][%
      \centering
      Paths with $\lambda = 0$\par
      Average length: 733.0\par
      Average check cost: 7219.5
   ]{
      \includegraphics[width=0.45\textwidth]
      {figs/bean-allpaths-lambda0.png}
   }
   % right side
   \subfloat[Paths with lambda=1][%
      \centering
      Paths with $\lambda = 1$\par
      Average length: 836.5\par
      Average check cost: 4692.6
   ]{
      \includegraphics[width=0.45\textwidth]
      {figs/bean-allpaths-lambda1.png}
   }
   \\
   \subfloat[Paths with lambda=0][%
      \centering
      Paths with $\lambda = 0$\par
      Average length: 733.0\par
      Average check cost: 2685.7
   ]{
      \includegraphics[width=0.45\textwidth]
      {figs/bean-allpaths-padded-lambda0.png}
   }
   % right side
   \subfloat[Paths with lambda=1][%
      \centering
      Paths with $\lambda = 1$\par
      Average length: 907.1\par
      Average check cost: 1064.5
   ]{
      \includegraphics[width=0.45\textwidth]
      {figs/bean-allpaths-padded-lambda1.png}
   }

   \caption{A simple 2D example of the Family PRM using
     a broad-phase check.
     Checking for collision with the grey box is 10x less expensive
     than with the actual black obstacle.}
     %\cdnote{I need to talk about this in the text.}}
   \label{fig:family:broad-phase-2d}
\end{figure}






\section{Application: Dynamic Environments}

\paragraph{Dynamic Environments.}
\label{subsec:family:dynamic-environments}

\begin{figure}
   \centering

   \begin{minipage}{.6\textwidth}
      \begin{tikzpicture}
      \tikzset{>=latex} % arrow heads
      \node[anchor=south west,inner sep=0] at (0,0)
        {\includegraphics[width=0.8\textwidth]{figs/chimp-voxels-delta.png}};

      \node[draw,inner sep=3pt,fill=white,fill opacity=0.9,align=center]
        (debrislab) at (0.7,1.0) {Debris object\\removed};
      \node[circle,inner sep=2,draw,fill=white] (debris) at (2.2,2.9) {};
      \draw[draw=black, double=white, double distance=1pt, line width=1pt]
        (debrislab.north) -- (debris);
        
      \node[draw,inner sep=3pt,fill=white,fill opacity=0.9,align=center]
        (addlab) at (5.0,2.0) {Additional\\voxels seen};
      \node[circle,inner sep=2,draw,fill=white] (added) at (4.4,5.0) {};
      \draw[draw=black, double=white, double distance=1pt, line width=1pt]
        (addlab.north) -- (added);
        
      \end{tikzpicture}
   \end{minipage}%
   \,%
   \begin{minipage}{.35 \textwidth}
      \includegraphics{build/retroactive-a}
      
      \includegraphics{build/retroactive-b}
   \end{minipage}

   \caption{
      Structured or unstructured dynamic environments
      can be represented as a family problem
      (see Section~\ref{subsec:family:dynamic-environments}).
      
      \vspace{0.05in}
      \noindent
      (Left) A disaster response robot maintains a
      dynamic unstructured environment model
      using coarse voxels
      (scene data from a debris-clearing task at a
      recent disaster response competition).
      Since the last planning query,
      voxels have been added (green) and removed (red).
      
      \vspace{0.05in}
      \noindent
      (Right) $\mathcal{C}$-subsets and relations
      can be added retroactively.
      Here, the graph for an initial query is checked w.r.t $S_1$.
      After environment changes,
      $S_1$ is redefined in terms of the $\mathcal{C}$-subsets
      derived from the set of common, added, and removed elements,
      allowing for reuse on a query in $S_2$.
      Here,
      the planner need only check existing path segments
      against added voxels in order to reuse them for the current query.}
   \label{fig:family:dynamic-environments}
\end{figure}

The sensors on most articulated robots allow them to maintain
dynamic environment models to track changing collision geometry.
These models might be
structured (e.g. recognizing objects with known models)
or unstructured (e.g. occupancy models).
In both cases,
even in a changing world,
there are often areas that are fixed between planning queries.

Prior work (e.g. \citep{jaillet2004dynamicprm})
leverages this by imposing a dichotomy between
\emph{fixed} and \emph{moving} components of $\mathcal{C}$.
Our formulation extends this to an arbitrary number of such labels,
including ones defined retroactively (i.e. during planning;
see Fig.~\ref{fig:family:dynamic-environments}).
By explicitly labeling such areas in workspace
(and leveraging the \emph{set containment} property
\citep{newmanbranicky1991cspacetransforms}),
we can represent this structure as a family motion planning problem.

\section{Other Stuff}

\pgfplotstableread{
barname mean error
RRTConnect 2.56378257751494 0.1819787187133847
LazyPRM(*) 2.59627093791926 0.07645708288005078
LEMUR(SQ) 1.95654549121872 0.04821756844687192
LEMUR(MQ) 1.5911829090115797 0.03471955055186312
}{\datalemurherbbin}

\begin{figure}
   \centering
   \begin{tikzpicture}
      \begin{axis}[
         width=6.0cm,
         height=4.0cm,
         xbar,
         bar width=12,
         xmin=0,
         enlarge y limits={abs=0.45cm},
         %xmin=0,xmax=95,
         %xtick pos=bottom,
         %symbolic y coords={E, F, R, A, B, W, P},
         yticklabels from table={\datalemurherbbin}{barname},
         ytick=data,
         ytick pos=left,
         xmajorgrids,
         %xmajorticks=false,
         ticklabel style={font=\footnotesize},
         xlabel={\footnotesize{Planning time $p$ (s)}},
         xlabel near ticks,
         ] 
      %\node[circle,fill=white,inner sep=1pt,text=black!40] at (axis cs:40,-5.6) {\scriptsize 40};
      %\node[circle,fill=white,inner sep=1pt,text=black!40] at (axis cs:60,-5.6) {\scriptsize 60};
      %\node[circle,fill=white,inner sep=1pt,text=black!40] at (axis cs:80,-5.6) {\scriptsize 80};
      \addplot[color=black,fill=black!20,error bars/.cd,x dir=both,x explicit]
         table[y expr=-\coordindex,x=mean,x error expr=\thisrow{error}]
         {\datalemurherbbin};
      \end{axis}
   \end{tikzpicture}
   \caption{LEMUR Herbbin, $\lambda = 1$}
\end{figure}

\pgfplotstableread{
barname mean error
RRTConnect 17.735649337767864 0.8572182241158215
LazyPRM(*) 11.02914777755716 0.2718847854283968
LEMUR(SQ) 8.987455253601558 0.22683157494511255
LEMUR(MQ) 5.995352478027379 0.23607184395127515
}{\datalemurworkcell}

\begin{figure}
   \centering
   \begin{tikzpicture}
      \begin{axis}[
         width=6.0cm,
         height=4.0cm,
         xbar,
         bar width=12,
         xmin=0,
         enlarge y limits={abs=0.45cm},
         %xmin=0,xmax=95,
         %xtick pos=bottom,
         %symbolic y coords={E, F, R, A, B, W, P},
         yticklabels from table={\datalemurworkcell}{barname},
         ytick=data,
         ytick pos=left,
         xmajorgrids,
         %xmajorticks=false,
         ticklabel style={font=\footnotesize},
         xlabel={\footnotesize{Planning time $p$ (s)}},
         xlabel near ticks,
         ] 
      \node[circle,fill=white,inner sep=1pt,text=black!40] at (axis cs:40,-5.6) {\scriptsize 40};
      \node[circle,fill=white,inner sep=1pt,text=black!40] at (axis cs:60,-5.6) {\scriptsize 60};
      \node[circle,fill=white,inner sep=1pt,text=black!40] at (axis cs:80,-5.6) {\scriptsize 80};
      \addplot[color=black,fill=black!20,error bars/.cd,x dir=both,x explicit]
         table[y expr=-\coordindex,x=mean,x error expr=\thisrow{error}]
         {\datalemurworkcell};
      \end{axis}
   \end{tikzpicture}
   \caption{LEMUR Workcell, $\lambda = 1$}
\end{figure}

% data over 84 data points
% there are 85 PlanToNamedConfiguration planning requests in
% my planning_queries/yaml/* directory;
% i paired them up, and these are the numbers across all pairs
\pgfplotstableread{
barname mean stderror95
FamilyFirst 0.3858953203473929 0.011569714605374383
FamilySecond 0.18286271038508334 0.0070175681823757274
}{\datalemurtableclear}

\begin{figure}
   \centering
   \begin{tikzpicture}
      \begin{axis}[
         width=6.0cm,
         height=3.0cm,
         xbar,
         bar width=12,
         xmin=0,
         enlarge y limits={abs=0.45cm},
         %xmin=0,xmax=95,
         %xtick pos=bottom,
         %symbolic y coords={E, F, R, A, B, W, P},
         yticklabels from table={\datalemurtableclear}{barname},
         ytick=data,
         ytick pos=left,
         xmajorgrids,
         %xmajorticks=false,
         ticklabel style={font=\footnotesize},
         xlabel={\footnotesize{Planning time $p$ (s)}},
         xlabel near ticks,
         ] 
      %\node[circle,fill=white,inner sep=1pt,text=black!40] at (axis cs:40,-5.6) {\scriptsize 40};
      %\node[circle,fill=white,inner sep=1pt,text=black!40] at (axis cs:60,-5.6) {\scriptsize 60};
      %\node[circle,fill=white,inner sep=1pt,text=black!40] at (axis cs:80,-5.6) {\scriptsize 80};
      \addplot[color=black,fill=black!20,error bars/.cd,x dir=both,x explicit]
         table[y expr=-\coordindex,x=mean,x error expr=\thisrow{stderror95}]
         {\datalemurtableclear};
      \end{axis}
   \end{tikzpicture}
   \caption{Family Tableclear, $\lambda = 0$}
\end{figure}
