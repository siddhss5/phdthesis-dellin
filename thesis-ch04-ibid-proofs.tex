\chapter{Appendix: IBiD Proofs}
\label{chap:appendix-ibid-proofs}


\begin{proof}[Proof of Theorem~\ref{thm:ibid-relaxation-notension}]
Consider any vertex $x$.
If $d^*(x) = \infty$,
then by (\ref{eqn:ibid-relaxation-props-nounder}) we must have that
$d(x) = \infty$.
Otherwise,
by (\ref{eqn:ibid-distance-function-global}),
there exists a path $p^*$ of length $d^*(x)$;
consider this path.
The first vertex on $p^*$ is $s$ with $d^*(s) = 0$,
and by (\ref{eqn:ibid-relaxation-props}),
$d(s) = d^*(s)$.
For each edge $e_{uv}$ on $p^*$ with $d(u) = d^*(u)$,
we will show that $d(v) = d^*(v)$.
By definition of the shortest path,
$d^*(u) + w(e_{uv}) = d^*(v)$.
Therefore $d(u) + w(e_{uv}) = d^*(v)$,
and by (\ref{eqn:ibid-relaxation-props-tens}),
we have $d^*(v) \geq d(v)$,
and by (\ref{eqn:ibid-relaxation-props-nounder})
we have $d^*(v) = d(v)$.
By induction along the path $p^*$,
we have that $d(x) = d^*(x)$.
\end{proof}


\begin{proof}[Proof of Theorem~\ref{thm:ibid-relaxation-sound}]
The proof proceeds as follows.
First, if $d' = \infty$,
then no edges can be in tension,
and so $d = d^*$ everywhere
as shown in Section~\ref{subsec:ibid-tension}.
Otherwise,
suppose that $d(x) \neq d^*(x)$
for some vertex $x$ with $d(x) \leq d'$.
By (\ref{eqn:ibid-relaxation-props}),
it would have to be that $d^*(x) < d(x)$.
Consider a true shortest path $p$ from $s$ to $x$;
by (\ref{eqn:ibid-relaxation-props})
such a path exists and has finite length $d^*(x)$.
By (\ref{eqn:ibid-relaxation-props}),
we have that $d^*(s) = d(s) = 0$
(and so $s$ and $x$ must be distinct).
Let $e_{uv}$ be the first edge along $p$ such that
$d^*(u) = d(u)$
but $d^*(v) < d(v)$.
Since $p$ is a shortest path,
edge $e_{uv}$ must therefore be in tension.
Since $w \geq 0$,
it must be that $d^*(u) \leq d^*(x)$;
further, since $d^*(u) = d(u)$,
$d^*(x) < d(x)$,
and $d(x) \leq d'$
it must be that $d(u) < d'$.
But then edge $e_{uv}$ is in tension with lower $d(u)$!
This contradiction implies that every vertex $x$
with $d(x) \leq d'$
must have $d(x) = d^*(x)$.
\end{proof}


\begin{proof}[Proof of Theorem~\ref{thm:ibid-relaxation-reconstruct}]
For any vertex $x$ with $d(x) \leq k$,
consider the following construction of a path from $s$ to $x$.
Initialize the path $p \leftarrow \{ x \}$;
note that by Theorem~\ref{thm:ibid-relaxation-sound},
the first vertex $v$ on $p$ has $d(v) = d^*(v)$.

At each iteration, terminate construction if $v=s$.
Otherwise,
let $u^*$ be the predecessor of $v$ which minimizes
$d(u) + w(e_{uv})$,
and prepend $u^*$ to $p$.
By (\ref{eqn:ibid-relaxation-props-nottoogood}),
it follows that $d(u^*) + w(e_{uv}^*) \leq d(v)$,
and since $d(v) = d^*(v)$ and
$d^*(v) \leq d^*(u) + w(e_{uv}^*)$
by definition of the distance function,
it follows that $d(u^*) \leq d^*(u^*)$.
In combination with (\ref{eqn:ibid-relaxation-props-nounder}),
this implies $d(u^*) = d^*(u^*)$,
and we can iterate.

The result of this construction is a path $p$ from $s$ to $x$
on which all vertices have $d(v) = d^*(v)$.
Therefore $p$ is a shortest path of length $d^*(x)$.
\end{proof}


\begin{proof}[Proof of Theorem~\ref{thm:ibid-bidir-sound}]
Note first that since $d_s(u) \leq D_s$,
by Theorem~\ref{thm:ibid-relaxation-sound},
$d_s(u) = d^*_s(u)$,
and so there exists a path from $s$ to $u$ of length $d_s(u)$.
Similarly, there exists a path from $v$ to $t$ of length $d_t(v)$,
and as a result,
there exists a path through $e^*_{uv}$ with length $\ell_e(e^*_{uv})$.

We will prove that this constitutes a shortest path by contradiction.
Suppose that a different shortest path $p'$ exists with
$\mbox{len}(p') < \ell_e(e^*_{uv})$.
Then it must also be that
$\mbox{len}(p') < D_s + D_t$.
Note that since $s \neq t$, $p'$ contains at least one edge.
We will consider two cases.

First, consider the case where $\mbox{len}(p') < D_s$,
so that $d_s^*(t) < D_s$.
In this case,
the last edge $e_{ut}'$ on $p'$
has $d_s^*(u') \leq d_s^*(t) < D_s$;
by Theorem~\ref{thm:ibid-relaxation-sound},
$u'$ therefore has $d_s(u') = d_s^*(u')$.
In addition,
since $D_t > 0$,
$t$ must be $t$-consistent with $d_t(t) = 0$.
Therefore,
it follows that $d_s^*(u') + w(e_{ut}') < d_s(u') + w(e_{ut}')$,
which contradicts the supposition.

In the second case with $D_s \leq \mbox{len}(p')$,
identify on $p'$ the edge $e'_{uv}$ adjoining the vertices $u'$, $v'$
such that $d_s^*(u') < D_s \leq d_s^*(v')$.
(Since $D_s > 0$, this edge will exist.)
Since $d_s^*(u') < D_s$,
by Theorem~\ref{thm:ibid-dynamicswsffp-sound},
$u'$ is therefore $s$-consistent with $d_s(u') = d_s^*(u')$.
Consider our supposition that
$d_s^*(u') + w(e_{uv}') + d_t^*(v') < D_s + D_t$.
Since $d_s^*(u') + w(e_{uv}') = d_s^*(v')$
and $D_s \leq d_s^*(v')$,
it follows that
$d_t^*(v') < D_t$.
Therefore,
by Theorem~\ref{thm:ibid-dynamicswsffp-sound},
$v'$ is $t$-consistent with $d_t(v') = d_t^*(v')$.
As a consequence,
the edge $e'_{uv}$ must be in $E_{\ms{conn}}$.
Therefore,
$d_s^*(u') + w(e_{uv}') + d_t^*(v')
   < d_s(u') + w(e_{uv}') + d_t(v')$,
which is a contradiction.
\end{proof}


\begin{proof}[Proof of Theorem~\ref{thm:ibid-dynamicswsffp-sound}]
The proof relies upon a path construction from $s$ to $x$
described by Lemma~\ref{lemma:ibid-dynamicswsffp-sound-conpath}.
Lemma~\ref{lemma:ibid-dynamicswsffp-sound-geq} then demonstrates
that the length of the path is $d(x)$,
and therefore that $d(x)$ can be no less than $d^*(x)$.
Finally,
Lemma~\ref{lemma:ibid-dynamicswsffp-sound-leq}
shows that $d(x)$ can be no greater than $d^*(x)$.
As a result,
the value $d(x)$ is correct,
and the path constructed
via Lemma~\ref{lemma:ibid-dynamicswsffp-sound-conpath}
is a shortest path.
\end{proof}

\begin{lemma}
For any consistent vertex $x$ with $d(x) \leq k_{\ms{min}}$,
there exists a path $p$ from $s$ to $x$
in which each vertex is consistent
and each edge $e_{uv}$ satisfies $d(u) + w(e_{uv}) = d(v)$.
\label{lemma:ibid-dynamicswsffp-sound-conpath}
\end{lemma}

\begin{proof}[Proof of Lemma~\ref{lemma:ibid-dynamicswsffp-sound-conpath}]
Construct the path $p$ as follows.
Initialize the path with the single vertex $x$.
Iteratively consider the first vertex $v$ on the path,
which is known to be consistent.
In the first case, if $v \neq s$,
then there exists a predecessor vertex $u$ and edge $e_{uv}$
with $d(u) + w(e_{uv}) = r(v)$.
Since $w > 0$ and $d(v) = r(v)$,
we have $d(u) < d(v) \leq d(x)$.
As a consequence,
$u$ is consistent;
prepend to the path the vertex $u$ and the edge $e_{uv}$,
and iterate.
In the second case, if $v = s$,
then we finish our construction of $p$.
Since the values $d(u)$ decrease monotonically
for all inserted vertices,
this process will terminate with a path $p$ beginning at $s$.
\end{proof}

\begin{lemma}
Any consistent vertex $x$ with $d(x) \leq k_{\ms{min}}$
has $d(x) \geq d^*(x)$.
\label{lemma:ibid-dynamicswsffp-sound-geq}
\end{lemma}

\begin{proof}[Proof of Lemma~\ref{lemma:ibid-dynamicswsffp-sound-geq}]
This follows directly from
Lemma~\ref{lemma:ibid-dynamicswsffp-sound-conpath}.
Since all vertices on the path are known consistent,
we must have $d(s) = 0$.
Further,
since the $d$-values across each edge in $p$
satisfy $d(u) + w(e_{uv}) = d(v)$,
it follows that $d(x) = \sum_{e \in p} w(e)$.
Therefore,
a path exists from $s$ to $x$ of length $d(x)$,
and so the true distance $d^*(x)$ must be upper-bounded by $d(x)$.
\end{proof}

\begin{lemma}
Any consistent vertex $v$ with $d(x) \leq k_{\ms{min}}$
has $d(x) \leq d^*(x)$.
\label{lemma:ibid-dynamicswsffp-sound-leq}
\end{lemma}

\begin{proof}[Proof of Lemma~\ref{lemma:ibid-dynamicswsffp-sound-leq}]
We demonstrate that $d(x) \leq d^*(x)$ by contradiction.
Suppose a vertex $x$ exists for which $d^*(x) < d(x)$.
Then there must exist a path $p'$ from $s$ to $x$ of length $d^*(x)$,
with $d^*(s) = 0$ and $d^*(u) + w(e_{uv}) = d^*(v)$ for each edge
in $p'$;
as a consequence,
we must have $d^*(v) < d(x)$ for all vertices $v$ on $p'$.
By Lemma~\ref{lemma:ibid-dynamicswsffp-sound-conpath},
we know that $s$ is consistent,
so $d(s) = 0$.
We will show that walking along
each edge $e_{uv}$ on $p'$ starting at $s$,
if $d(u) \leq d^*(u)$,
then $d(v) \leq d^*(v)$.

By definition,
we have $d(u) + w(e_{uv}) \geq r(v)$,
so that $d(u) - d^*(u) \geq r(v) - d^*(v)$.
Therefore,
it follows that $r(v) \leq d^*(v)$.
Since $k(v) \leq d^*(v)$,
it follows that $v$ is consistent,
so $d(v) \leq d^*(v)$.
We can replicate this logic down the path.
As a result,
it follows that $d(x) \leq d^*(x)$.
But this contraducts our supposition that $d^*(x) < d(x)$,
and therefore such a vertex $x$ cannot exist.
\end{proof}


\begin{proof}[Proof of Theorem~\ref{thm:ibid-sound}]
We will prove this by contradiction.
Suppose that a path $p'$ exists with
$\mbox{len}(p') < \min_{e \in E_{\ms{conn}}} \left( d_s(u) + w(e_{uv}) + d_t(v) \right)$.
Then it must also be that
$\mbox{len}(p') < K_s + K_t$.
We will consider two cases.

First, consider the case where $K_s > \mbox{len}(p')$,
so that $K_s > d_s^*(t)$.
In this case,
the last edge $e_{ut}'$ on $p'$
has $d_s^*(u') < d_s^*(t) < K_s$;
by Theorem~\ref{thm:ibid-dynamicswsffp-sound},
$u'$ is therefore $s$-consistent with $d_s(u') = d_s^*(u')$.
In addition,
since $K_t > 0$,
$t$ must be $t$-consistent with $d_t(t) = 0$.
Therefore,
it follows that $d_s^*(u') + w(e_{ut}') < d_s(u') + w(e_{ut}')$,
which contradicts the supposition.

In the second case with $K_s \leq \mbox{len}(p')$,
identify on $p'$ the edge $e'_{uv}$ adjoining the vertices $u'$, $v'$
such that $d_s^*(u') < K_s \leq d_s^*(v')$.
(Since $K_s > 0$, this edge will exist.)
Since $d_s^*(u') < K_s$,
by Theorem~\ref{thm:ibid-dynamicswsffp-sound},
$u'$ is therefore $s$-consistent with $d_s(u') = d_s^*(u')$.
Consider our supposition that
$d_s^*(u') + w(e_{uv}') + d_t^*(v') < K_s + K_t$.
Since $d_s^*(u') + w(e_{uv}') = d_s^*(v')$
and $K_s \leq d_s^*(v')$,
it follows that
$d_t^*(v') < K_t$.
Therefore,
by Theorem~\ref{thm:ibid-dynamicswsffp-sound},
$v'$ is $t$-consistent with $d_t(v') = d_t^*(v')$.
As a consequence,
the edge $e'_{uv}$ must be in $E_{\ms{conn}}$.
Therefore,
$d_s^*(u') + w(e_{uv}') + d_t^*(v')
   < d_s(u') + w(e_{uv}') + d_t(v')$,
which is a contradiction.
\end{proof}
