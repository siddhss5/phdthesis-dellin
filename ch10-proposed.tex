\chapter[Summary of Proposed Work]{Summary of\\Proposed Work}

\section{Research Questions}
\label{sec:research-questions}

List of research questions.

\begin{center}
\begin{tabular}{llc}
\toprule
   \multicolumn{2}{l}{Research Question}
      & Chapter \\
\midrule
   \ref{ques:choosing-lambda}
      &
      \begin{minipage}[c]{0.8\columnwidth}%
      \nameref{ques:choosing-lambda}
      \end{minipage}%
      & \ref{chap:inflate} \\[10pt]
   \ref{ques:incremental-search}
      &
      \begin{minipage}[c]{0.8\columnwidth}%
      \nameref{ques:incremental-search}
      \end{minipage}%
      & \ref{chap:inflate} \\[10pt]
   \ref{ques:batching}
      &
      \begin{minipage}[c]{0.8\columnwidth}%
      \nameref{ques:batching}
      \end{minipage}%
      & \ref{chap:graphs-in-continuous} \\[10pt]
   \ref{ques:multi-set-suited}
      &
      \begin{minipage}[c]{0.8\columnwidth}%
      \nameref{ques:multi-set-suited}
      \end{minipage}%
      & \ref{chap:multi-set-prm} \\[10pt]
   \ref{ques:how-sequence}
      &
      \begin{minipage}[c]{0.8\columnwidth}%
      \nameref{ques:how-sequence}
      \end{minipage}%
      & \ref{chap:task-planning} \\[10pt]
   \ref{ques:drc-compare}
      &
      \begin{minipage}[c]{0.8\columnwidth}%
      \nameref{ques:drc-compare}
      \end{minipage}%
      & \ref{chap:task-planning} \\[10pt]
   \ref{ques:herb-performance}
      &
      \begin{minipage}[c]{0.8\columnwidth}%
      \nameref{ques:herb-performance}
      \end{minipage}%
      & \ref{chap:task-planning} \\[6pt]
\bottomrule
\end{tabular}
\end{center}

{
%\renewcommand\thesubsection{\thesection.Q\arabic{subsection}}
\renewcommand\thesubsection{Q\arabic{subsection}}

\subsection{How should the $\lambda$ parameter mediating
   between planning and execution cost be chosen?}
\label{ques:choosing-lambda}

See Chapter~\ref{chap:inflate}.

\subsection{How can incremental graph search ideas (e.g. LPA*)
   be used to efficiently implement the $\mbox{E}^8$ algorithm?}
\label{ques:incremental-search}

See Chapter~\ref{chap:inflate}.

\subsection{What is the best way to handle batching?}
\label{ques:batching}

Chapter~\ref{chap:graphs-in-continuous}
discusses how to embed roadmaps in $\mathcal{C}$
so that they can be searched by $\mbox{E}^8$.

The problem with na\"{\i}vly running $\mbox{E}^8$ on a
dense roadmap in $\mathcal{C}$
is that it tends to bunch up in local minima.
This is because reducing the continuous planning problem
to a graph search ignores the spatial correlation
inherent in $\mathcal{C}_{\mbox{\scriptsize free}}$.

One way to capture this is to maintain a probabalistic model
of $\mathcal{C}_{\mbox{\scriptsize free}}$,
and then optimize in expectation.
In particular,
instead of greedily choosing the best path based on
optimistic estimates of one-time planning and execution cost:
\begin{equation}
   f(\pi) = \lambda \hat{f}_p(\pi) + (1-\lambda) \hat{f}_x(\pi),
\end{equation}
we instead reason over the total \emph{expected} remaining cost:
\begin{align}
   f(\pi)
      &= E \left[ \lambda f_p(\pi) + (1-\lambda) f_x(\pi) \right] \\
   &= P_{\mbox{\scriptsize free}}(\pi)
      \left[ \lambda \hat{f}_p(\pi) + (1-\lambda) \hat{f}_x(\pi) \right]
      + (1-P_{\mbox{\scriptsize free}}(\pi))
      \left[ \lambda F_p + (1-\lambda) F_x \right]
\end{align}

Consider the the problem from Figure~\ref{fig:example-in-expectation}.
There are an infinite number of paths to the goal,
each consisting of walking along the sidewalk,
followed by crossing the street perpendicuarly at a particular
position $x$.
The sidewalk is known to be collision-free,
whereas each position on the street must be tested for collision
with obstacles with planning validation cost $\hat{f}_p(\pi)$
independent of $x$.
Execution cost $f_x(\pi)$ is given by $|x|+c$.

Suppose we first test walking straight across the street $\pi_0$
(knowing nothing, this is clearly the optimistically cheapest path)
and this is deemed in collision.
Which path should we consider next (e.g $\pi_a$ or $\pi_b$)?

What is our model for $P_{\mbox{\scriptsize free}}(\pi)$?
Radial basis functions.

We are operating under assumptions:
\begin{itemize}
\item Single-shot greedy (won't choose \emph{sets} of paths
   which minimize remaining effort)
\item Operates over \emph{paths} instead of configurations
   or edges (won't probe points, no explicit exploration)
\end{itemize}

\begin{figure}
   \begin{center}
   \includegraphics{build/example-in-expectation}
   \end{center}
   \caption{Simple example problem to illustrate optimizing
      remaining ensemble cost in expectation.}
   \label{fig:example-in-expectation}
\end{figure}

\subsection{What types of multi-set structure is the Multi-Set PRM
   well-suited to, and what not?}
\label{ques:multi-set-suited}

See Chapter~\ref{chap:multi-set-prm}.

\subsection{How should the \textsc{Proteus} task planner
   sample roots in multi-step problems?}
\label{ques:how-sequence}

See Chapter~\ref{chap:task-planning}.

\subsection{How does the \textsc{Proteus} task planner
   compare experimentally to the BiRRT used at the DRC Trials?}
\label{ques:drc-compare}

See Chapter~\ref{chap:task-planning}.

\subsection{How does the \textsc{Proteus} task planner
   perform on everyday kitchen tasks by the \textsc{Herb} robot?}
\label{ques:herb-performance}

See Chapter~\ref{chap:task-planning}.

}%for custom Q1 subsection numbering

\section{Timeline}

See this table for the timeline.

\begin{center}
\begin{tabular}{lll}
\hline
Topic & Sec. & Deadline \\
\hline
Greedy PRM, Multi-Set Planning & & January 2015 (RSS) (completed) \\
Proposal & & February 2015 \\
The Greedy PRM: Optimizing Total Task Cost & & March 2015 (IROS) \\
DRC Integration & & April-May 2015 \\
Multi-Set Planning at the DRC & & June 2015 (Humanoids) \\
Bored Robots: Hypothesized Conservative Volumes & & October 2015 (ICRA) \\
Writing & & November 2015 \\
Defence & & December 2015 \\
\hline
\end{tabular}
\end{center}

\section{Open-Source Software}

\begin{itemize}
\item Release implementation of checkmask PRM (OMPL)
\item Release implementation of multi-set decomposer (OpenRAVE+OMPL)
\item Re-release orcdchomp with permissive license?
\end{itemize}
