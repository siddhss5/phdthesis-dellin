\chapter{Appendix: LazySP Proofs}
\label{sec:appendix-proofs}

\subsection{LazySP}

\begin{proof}[Proof of Theorem \ref{thm:lazy-optimality}]
Let $p^*$ be an optimal path w.r.t. $w$,
with $\ell^* = \mbox{len}(p^*, w)$.
Since $w_{\ms{est}}(e) \leq \epsilon \, w(e)$ and $\epsilon \geq 1$,
it follows that regardless of which edges are stored in $W_{\ms{eval}}$,
$w_{\ms{lazy}}(e) \leq \epsilon \, w(e)$,
and therefore
$\mbox{len}(p^*, w_{\ms{lazy}}) \leq \epsilon \, \ell^*$.
Now,
since the inner \textsc{ShortestPath} algorithm terminated with
$p_{\ms{ret}}$,
we know that
$\mbox{len}(p_{\ms{ret}}, w_{\ms{lazy}}) \leq \mbox{len}(p^*, w_{\ms{lazy}})$.
Further,
since the algorithm terminated with $p_{\ms{ret}}$,
each edge on $p_{\ms{ret}}$ has been evaluated;
therefore,
$\mbox{len}(p_{\ms{ret}}, w) = \mbox{len}(p_{\ms{ret}}, w_{\ms{lazy}})$.
Therefore,
$\mbox{len}(p_{\ms{ret}}, w) \leq \epsilon \, \ell^*$.
\end{proof}

\begin{proof}[Proof of Theorem \ref{thm:lazy-completeness}]
In this case,
the algorithm will evaluate at least unevaluated edge at
each iteration.
Since there are a finite number of edges,
eventually the algorithm will terminate.
\end{proof}

\subsection{A* Equivalence}

\begin{proof}[Proof of Invariant \ref{inv:astar-cundisc-popen}]
If $v$ is discovered, then it must either be on OPEN or CLOSED.
$v$ can be on CLOSED only after it has been expanded,
in which case $s$ would be discovered (which it is not).
Therefore, $v$ must be on OPEN.
\end{proof}

\begin{proof}[Proof of Invariant \ref{inv:astar-wless-popen}]
Clearly the invariant holds at the beginning of the algorithm,
with only $v_{\ms{start}}$ on OPEN.
If the invariant were to no longer hold after some iteration,
then there must exist some pair of discovered vertices $v$ and $v'$
with $v$ on CLOSED and $g[v] + w(v,v') < g[v']$.
Since $v$ is on CLOSED,
it must have been expanded at some previous iteration,
immediately after which the inequality could not have held
because $g[v']$ is updated upon expansion of $v$.
Therefore,
the inequality must have newly held after some intervening iteration,
with $v$ remaining on CLOSED.
Since the values $g$ are monotonically non-increasing and $w$ is fixed,
this implies that $g[v]$ must have been updated (lower).
However,
if this had happened,
then $v$ would have been removed from CLOSED and placed on OPEN.
This contradiction implies that the invariant holds at every iteration.
\end{proof}

\begin{proof}[Proof of Theorem \ref{thm:astar-equiv-from-lazy}]
Consider path $p^*_{\ms{lazy}}$ with length $\ell^*_{\ms{lazy}}$
yielding frontier vertex $v_{\ms{frontier}}$ via \textsc{SelectExpand}.
Construct a vertex sequence $s$ as follows.
Initialize $s$ with the vertices on $p^*_{\ms{lazy}}$
from $v_{\ms{start}}$ to $v_{\ms{frontier}}$, inclusive.
Let $N$ be the number of consecutive vertices at the start of $s$
for which $f(v) = \ell^*_{\ms{lazy}}$
(Note that the first vertex on $p^*_{\ms{lazy}}$, $v_{\ms{start}}$,
must have $f(v_{\ms{start}}) = \ell^*_{\ms{lazy}}$,
so $N \geq 1$.)
Remove from the start of $s$ the first $N-1$ vertices.
Note that at most the first vertex on $s$ has
$f(v) = \ell^*_{\ms{lazy}}$,
and the last vertex on $s$ must be $v_{\ms{frontier}}$.

Now we show that each vertex in this sequence $s$,
considered by A* in turn,
exists on OPEN with minimal $f$-value.
Iteratively consider the following procedure for sequence $s$.
Throughout,
we know that there must not be any vertex with
$f(v) < \ell^*_{\ms{lazy}}$,
that would imply that a different path through $v_b$ shorter than
$\ell^*_{\ms{lazy}}$ exists,
in which case $p^*_{\ms{lazy}}$ could not have been chosen.

If the sequence has length $>1$,
then consider the first two vertices on $s$, $v_a$ and $v_b$.
By construction,
$f(v_a) = \ell^*_{\ms{lazy}}$
and 
$f(v_b) \neq \ell^*_{\ms{lazy}}$.
In fact, from above
we know that $f(v_b) > \ell^*_{\ms{lazy}}$.
Therefore,
we have that $f(v_a) < f(v_b)$,
therefore and $g[v_a] + w(v_a,v_b) < g[v_b]$.
By Invariant~\ref{inv:astar-wless-popen},
$v_a$ must be on OPEN,
and with $f(v_a) = \ell^*_{\ms{lazy}}$,
it can therefore be considered by A*.
After it is expanded,
$f(v_b) = \ell^*_{\ms{lazy}}$,
and we can repeat the above procedure
with the sequence formed by removing the $v_a$ from $s$.

If instead the sequence has length $1$,
then it must be exactly $(v_{\ms{frontier}})$,
with $f(v_{\ms{frontier}}) = \ell^*_{\ms{lazy}}$.
Since the edge after $f(v_{\ms{frontier}})$ is not
evaluated,
then by Invariant~\ref{inv:astar-cundisc-popen},
$v_{\ms{frontier}}$ must be on OPEN,
and will therefore be expanded next.
\end{proof}

\begin{proof}[Proof of Theorem \ref{thm:astar-equiv-to-lazy}]
Given that all vertices in $s_{\ms{candidate}}$ besides the last
are re-expansions,
they can be expanded with no edge evaluations.
Once the last vertex,
$v_{\ms{frontier}}$,
is to be expanded by A*,
suppose it has $f$-value $\ell$.

First,
we will show that there exists a path with length $\ell$ w.r.t.
$w_{\ms{lazy}}$
wherein all edges before $v_{\ms{frontier}}$ have been evaluated,
and the first edge after $v_{\ms{frontier}}$ has not.
Let $p_a$ be a shortest path from $v_{\ms{start}}$
to $v_{\ms{frontier}}$ consisting of only evaluated edges.
The length of this $p_a$ must be equal to $g[v_{\ms{frontier}}]$;
if it were not,
there would be some previous vertex on $p_a$ with lower $f$-value
than $v_{\ms{frontier}}$,
which would necessarily have been expanded first.
Let $p_b$ be the a shortest path from $v_{\ms{frontier}}$
to $v_{\ms{goal}}$.
The length of $p_b$ must be $h_{\ms{lazy}}(v_{\ms{frontier}})$
by definition.
Therefore, the path $(p_a, p_b)$ must have length $\ell$,
and since $v_{\ms{frontier}}$ is a new expansion,
the first edge on $p_b$ must be unevaluated.

Second,
we will show that there does not exist any path shorter than $\ell$
w.r.t. $w_{\ms{lazy}}$.
Suppose $p'$ were such a path, with length $\ell' < \ell$.
Clearly, $v_{\ms{start}}$ would have $f$-value $\ell'$ (although
it may not be on OPEN).
Consider each pair of vertices $(v_a, v_b)$ along $p'$ in turn.
In each case,
if $v_b$ were either undiscovered,
or if $g[v_a] + w(v_a, v_b) < g[v_b]$,
then $v_a$ would be on OPEN
(via Invariants \ref{thm:lazy-completeness}
and \ref{inv:astar-wless-popen}, respectively)
with $f(v_a) = \ell'$,
and would therefore have been expanded before $v_{\ms{frontier}}$.
Otherwise,
we know that $f(v_b) = \ell'$,
and we can continue to the next pair on $p'$.
\end{proof}

\subsection{LWA* Equivalence}

\begin{proof}[Proof of Invariant \ref{inv:lwastar}]
Clearly the invariant holds at the beginning of the algorithm
with only $g[v_{\ms{start}}] = 0$,
since the inequality holds only for the out-edges of $v_{\ms{start}}$,
with $v_{\ms{start}}$ on $Q_v$.
Consider each subsequent iteration.
If a vertex $v$ is popped from $Q_v$,
then this may invalidate the invariant for all successors of $v$;
however,
since all out-edges are immediately added to $Q_e$,
the invariant must hold.
Consider instead if an edge $(v, v')$ which satisfies the inequality
is popped from $Q_e$.
Due to the inequality,
we know that $g[v']$ will be recalculated as
$g[v'] = g[v] + w(v,v')$,
so that the inequality is no longer satisfied for edge $(v,v')$.
However,
reducing the value $g[v']$ may introduce satisfied inequalities across
subsequent out-edges of $v'$,
but since $v'$ is added to $Q_v$,
the invariant continues to hold.
\end{proof}

\begin{proof}[Proof of Theorem \ref{thm:lwastar-equiv-from-lazy}]
In the first component of the equivalence,
we will show that for any path $p$ minimizing $w_{\ms{lazy}}$
allowable to LazySP-Forward,
with $(v_a, v_b)$ the first unevaluated edge on $p$,
there exists a sequence of vertices and edges on
$Q_v$ and $Q_e$ allowable to LWA*
such that edge $(v_a, v_b)$ is the first to be newly evaluated.
Let the length of $p$ w.r.t. $w_{\ms{lazy}}$ be $\ell$.'

We will first show that no vertex on $Q_v$ or edge on $Q_e$
can have $f(\cdot) < \ell$.
Suppose such a vertex $v$, or edge $e$ with source vertex $v$,
exists.
Then $g[v] + h(v) < \ell$,
and there must be some path $p'$ consisting of an evaluated segment
from $v_{\ms{start}}$ to $v$ of length $g[v]$,
followed a segment from $v$ to $v_{\ms{goal}}$ of length $h(v)$.
But then this path should have been chosen by LazySP.

Next, we will show a procedure for generating an allowable
sequence for LWA*.
We will iteratively consider a sequence of path segments,
starting with the segment from $v_{\ms{start}}$ to $v_a$,
and becoming progressively shorter at each iteration by removing the
first vertex and edge on the path.
We will show that the first vertex on each segment $v_f$
has $g[v_f] = \ell - h(v_f)$.
By definition,
this is true of the first such segment, since $g[v_{\ms{start}}] = 0$.
For each but the last such segment,
consider the first edge, $(v_f, v_s)$.
If $v_s$ has the correct $g[\cdot]$,
we can continue to the next segment immediately.
Otherwise,
either $v_f$ is on $Q_v$ or $(v_f, v_s)$ is on $Q_e$ by
Invariant~\ref{inv:lwastar}.
If the former is true,
then $v_f$ can be popped from $Q_v$ with $f = \ell$,
thereby adding $(v_f, v_s)$ to $Q_e$.
Then,
$(v_f, v_s)$ can be popped from $Q_e$ with $f = \ell$,
resulting in $g[v_s] = \ell - h(v_s)$.
We can then move on to the next segment.

At the end of this process,
we have the trivial segment $(v_a)$,
with $g[v_a] = \ell - h(v_a)$.
If $v_a$ is on $Q_v$, then pop it (with $f(v_a) = \ell$),
placing $e_{ab}$ on $Q_e$;
otherwise, since $e_{ab}$ is unevaluated,
it must already be on $Q_e$.
Since $f(e_{ab}) = \ell$, we can pop and evaluate it.
\end{proof}

\begin{proof}[Proof of Theorem \ref{thm:lwastar-equiv-to-lazy}]
Given that all vertices in $s_{\ms{candidate}}$ entail no edge evaluations,
and all edges therein are re-expansions,
they can be considered with no edge evaluations.
Once the last edge $e_{xy}$ is to be expanded by LWA*,
suppose it has $f$-value $\ell$.

First,
we will show that there exists a path with length $\ell$ w.r.t.
$w_{\ms{lazy}}$
which traverses unevaluated edge $e_{xy}$
wherein all edges before $v_x$ have been evaluated.
Let $p_x$ be a shortest path segment from $v_{\ms{start}}$
to $v_x$ consisting of only evaluated edges.
The length of $p_x$ must be equal to $g[v_x]$;
if it were not,
there would be some previous vertex on $p_x$ with lower $f$-value
than $v_x$,
which would necessarily have been expanded first.
Let $p_y$ be the a shortest path from $v_y$
to $v_{\ms{goal}}$.
The length of $p_y$ must be $h_{\ms{lazy}}(v_y)$
by definition.
Therefore, the path $(p_x, e_{xy}, p_y)$ must have length $\ell$.

Second,
we will show that there does not exist any path shorter than $\ell$
w.r.t. $w_{\ms{lazy}}$.
Suppose $p'$ were such a path, with length $\ell' < \ell$,
and with first unevaluated edge $e'_{xy}$.
Clearly, $v_{\ms{start}}$ has
$g[v_{\ms{start}}] = \ell' - h(v_{\ms{start}})$.
Consider each evaluated edge $e'_{ab}$ along $p'$ in turn.
In each case,
if $g[v'_b] \neq \ell' - h(v'_b)$,
then either $v'_a$ or $e'_{ab}$ would be on $Q_v$ or $Q_e$
with $f(\cdot) = \ell'$,
and would therefore be expanded before $e_{xy}$.
Therefore,
$e'_{xy}$ would then be popped from $Q_e$ with $f(e'_{xy}) = \ell'$,
and it would have been evaluated before $e_{xy}$ with $f(e_{xy}) = \ell$.
\end{proof}

\chapter{Appendix: LazySP Timing Results}
\label{sec:appendix-timing}

We include an accounting of the cumulative computation time taken by
each component of LazySP for each of the seven selectors
for each problem class (Figure~\ref{fig:table-timing-results}).

\begin{figure*}
   \begin{widepage}
   \centering
   {\small%
   \setlength{\tabcolsep}{0.06cm}%
   \begin{tabular}{l@{\hskip 9pt}rc@{\hskip 0pt}r@{\hskip 9pt}rc@{\hskip 0pt}r@{\hskip 9pt}rc@{\hskip 0pt}r@{\hskip 9pt}rc@{\hskip 0pt}r@{\hskip 9pt}rc@{\hskip 0pt}r@{\hskip 9pt}rc@{\hskip 0pt}r@{\hskip 9pt}rc@{\hskip 0pt}r}
      \toprule
         & \multicolumn{3}{c}{Expand} & \multicolumn{3}{c}{Forward} & \multicolumn{3}{c}{Reverse} & \multicolumn{3}{c}{Alternate}
         & \multicolumn{3}{c}{Bisect} & \multicolumn{3}{c}{WeightSamp} & \multicolumn{3}{c}{Partition} \\
      \midrule
      \addlinespace[0.5em]
      PartConn & & & & & & & & & & & & & & & & & & & & & \\
      \;\;\emph{total (ms)}    &\bf1.22 &$\pm$&   0.04 &    1.96 &$\pm$&  0.06 &    1.86 &$\pm$&  0.06 &\bf1.20 &$\pm$& 0.03 &  2.41 &$\pm$& 0.06 & 4807.19 &$\pm$& 135.22 &   15.81 &$\pm$&  0.16 \\
      \;\;\emph{sel-init (ms)} & --\;\; &     &        &  --\;\; &     &       &  --\;\; &     &       & --\;\; &     &      &--\;\; &     &      &  --\;\; &     &        &   12.49 &$\pm$&  0.11 \\
      \;\;\emph{online (ms)}   &\bf1.22 &$\pm$&   0.04 &    1.96 &$\pm$&  0.06 &    1.86 &$\pm$&  0.06 &\bf1.20 &$\pm$& 0.03 &  2.41 &$\pm$& 0.06 & 4807.19 &$\pm$& 135.22 &    3.32 &$\pm$&  0.10 \\
      \;\;\emph{search (ms)}   &   0.48 &$\pm$&   0.01 &    1.12 &$\pm$&  0.03 &    1.05 &$\pm$&  0.03 &   0.68 &$\pm$& 0.02 &  1.38 &$\pm$& 0.04 &    0.70 &$\pm$&   0.02 &    0.68 &$\pm$&  0.02 \\
      \;\;\emph{sel (ms)}      &   0.02 &$\pm$&   0.00 &    0.01 &$\pm$&  0.00 &    0.01 &$\pm$&  0.00 &   0.01 &$\pm$& 0.00 &  0.03 &$\pm$& 0.00 & 4805.64 &$\pm$& 135.18 &    2.07 &$\pm$&  0.06 \\
      \;\;\emph{eval (ms)}     & --\;\; &     &        &  --\;\; &     &       &  --\;\; &     &       & --\;\; &     &      &--\;\; &     &      &  --\;\; &     &        &  --\;\; &     &       \\
      \;\;\emph{eval (edges)}  &  87.10 &$\pm$&   2.39 &   35.86 &$\pm$&  1.04 &   34.84 &$\pm$&  1.04 &  22.23 &$\pm$& 0.60 & 44.81 &$\pm$& 1.11 &\bf20.66 &$\pm$&   0.57 &\bf20.39 &$\pm$&  0.56 \\
      \addlinespace[0.5em]
      UnitSquare & & & & & & & & & & & & & & & & & & & & & \\
      \;\;\emph{total (ms)}    &\bf0.91 &$\pm$&   0.03 &    1.47 &$\pm$&  0.06 &    1.49 &$\pm$&  0.06 &\bf0.94 &$\pm$& 0.03 &  1.71 &$\pm$& 0.04 & 3864.95 &$\pm$& 117.66 &   15.13 &$\pm$&  0.14 \\
      \;\;\emph{sel-init (ms)} & --\;\; &     &        &  --\;\; &     &       &  --\;\; &     &       & --\;\; &     &      &--\;\; &     &      &  --\;\; &     &        &   13.41 &$\pm$&  0.12 \\
      \;\;\emph{online (ms)}   &\bf0.91 &$\pm$&   0.03 &    1.47 &$\pm$&  0.06 &    1.49 &$\pm$&  0.06 &\bf0.94 &$\pm$& 0.03 &  1.71 &$\pm$& 0.04 & 3864.95 &$\pm$& 117.66 &    1.72 &$\pm$&  0.06 \\
      \;\;\emph{search (ms)}   &   0.35 &$\pm$&   0.01 &    0.79 &$\pm$&  0.03 &    0.82 &$\pm$&  0.03 &   0.51 &$\pm$& 0.02 &  0.92 &$\pm$& 0.02 &    0.75 &$\pm$&   0.02 &    0.45 &$\pm$&  0.01 \\
      \;\;\emph{sel (ms)}      &   0.01 &$\pm$&   0.00 &    0.01 &$\pm$&  0.00 &    0.01 &$\pm$&  0.00 &   0.01 &$\pm$& 0.00 &  0.02 &$\pm$& 0.00 & 3863.49 &$\pm$& 117.62 &    0.87 &$\pm$&  0.03 \\
      \;\;\emph{eval (ms)}     & --\;\; &     &        &  --\;\; &     &       &  --\;\; &     &       & --\;\; &     &      &--\;\; &     &      &  --\;\; &     &        &  --\;\; &     &       \\
      \;\;\emph{eval (edges)}  &  69.21 &$\pm$&   2.55 &   27.29 &$\pm$&  1.03 &   27.69 &$\pm$&  1.02 &  17.82 &$\pm$& 0.60 & 32.62 &$\pm$& 0.72 &   15.58 &$\pm$&   0.47 &\bf14.08 &$\pm$&  0.46 \\
      \addlinespace[0.5em]
      ArmPlan (avg) & & & & & & & & & & & & & & & & & & & & & \\
      \;\;\emph{total (s)}    &  269.82 &$\pm$&  17.95 &\bf 5.90 &$\pm$&  0.46 &    8.22 &$\pm$&  0.53 &\bf5.96 &$\pm$& 0.31 &  7.34 &$\pm$& 0.43 & 3402.21 &$\pm$& 172.20 &  496.57 &$\pm$&  5.53 \\
      \;\;\emph{sel-init (s)} &  --\;\; &     &        &  --\;\; &     &       &  --\;\; &     &       & --\;\; &     &      &--\;\; &     &      &  --\;\; &     &        &  490.77 &$\pm$&  5.51 \\
      \;\;\emph{online (s)}   &  269.82 &$\pm$&  17.95 &\bf 5.90 &$\pm$&  0.46 &    8.22 &$\pm$&  0.53 &\bf5.96 &$\pm$& 0.31 &  7.34 &$\pm$& 0.43 & 3402.21 &$\pm$& 172.20 &\bf 5.80 &$\pm$&  0.28 \\
      \;\;\emph{search (s)}   &    0.02 &$\pm$&   0.00 &    0.02 &$\pm$&  0.00 &    0.02 &$\pm$&  0.00 &   0.02 &$\pm$& 0.00 &  0.02 &$\pm$& 0.00 &    0.02 &$\pm$&   0.00 &    0.04 &$\pm$&  0.00 \\
      \;\;\emph{sel (s)}      &    0.00 &$\pm$&   0.00 &    0.00 &$\pm$&  0.00 &    0.00 &$\pm$&  0.00 &   0.00 &$\pm$& 0.00 &  0.00 &$\pm$& 0.00 & 3392.76 &$\pm$& 171.74 &    1.54 &$\pm$&  0.09 \\
      \;\;\emph{eval (s)}     &  269.78 &$\pm$&  17.95 &    5.87 &$\pm$&  0.45 &    8.20 &$\pm$&  0.52 &   5.94 &$\pm$& 0.31 &  7.31 &$\pm$& 0.43 &    9.39 &$\pm$&   0.57 &    4.21 &$\pm$&  0.22 \\
      \;\;\emph{eval (edges)} &  949.05 &$\pm$&  63.46 &   63.62 &$\pm$&  4.15 &   74.94 &$\pm$&  5.07 &  55.48 &$\pm$& 2.95 & 68.01 &$\pm$& 3.86 &   56.93 &$\pm$&   3.37 &\bf48.07 &$\pm$&  2.44 \\
      \addlinespace[0.25em]
      ArmPlan1 & & & & & & & & & & & & & & & & & & & & & \\
      \;\;\emph{total (s)}    &  109.09 &$\pm$&  14.15 &\bf 4.81 &$\pm$&  0.49 &   14.81 &$\pm$&  1.45 &   7.03 &$\pm$& 0.63 &  7.91 &$\pm$& 0.70 & 3375.35 &$\pm$& 319.81 &  496.74 &$\pm$&  8.22 \\
      \;\;\emph{sel-init (s)} &  --\;\; &     &        &  --\;\; &     &       &  --\;\; &     &       & --\;\; &     &      &--\;\; &     &      &  --\;\; &     &        &  489.49 &$\pm$&  8.18 \\
      \;\;\emph{online (s)}   &  109.09 &$\pm$&  14.15 &\bf 4.81 &$\pm$&  0.49 &   14.81 &$\pm$&  1.45 &   7.03 &$\pm$& 0.63 &  7.91 &$\pm$& 0.70 & 3375.35 &$\pm$& 319.81 &    7.25 &$\pm$&  0.66 \\
      \;\;\emph{search (s)}   &    0.02 &$\pm$&   0.00 &    0.02 &$\pm$&  0.00 &    0.03 &$\pm$&  0.00 &   0.02 &$\pm$& 0.00 &  0.02 &$\pm$& 0.00 &    0.02 &$\pm$&   0.00 &    0.04 &$\pm$&  0.00 \\
      \;\;\emph{sel (s)}      &    0.00 &$\pm$&   0.00 &    0.00 &$\pm$&  0.00 &    0.00 &$\pm$&  0.00 &   0.00 &$\pm$& 0.00 &  0.00 &$\pm$& 0.00 & 3358.82 &$\pm$& 318.17 &    1.61 &$\pm$&  0.16 \\
      \;\;\emph{eval (s)}     &  109.07 &$\pm$&  14.15 &    4.78 &$\pm$&  0.49 &   14.77 &$\pm$&  1.44 &   7.01 &$\pm$& 0.63 &  7.88 &$\pm$& 0.70 &   16.47 &$\pm$&   1.68 &    5.59 &$\pm$&  0.51 \\
      \;\;\emph{eval (edges)} &  344.74 &$\pm$&  39.63 &\bf49.72 &$\pm$&  4.25 &   95.58 &$\pm$&  9.67 &  59.44 &$\pm$& 5.06 & 58.90 &$\pm$& 4.74 &   73.72 &$\pm$&   7.63 &\bf50.66 &$\pm$&  4.43 \\
      \addlinespace[0.25em]
      ArmPlan2 & & & & & & & & & & & & & & & & & & & & & \\
      \;\;\emph{total (s)}    &  166.19 &$\pm$&   9.29 &\bf 3.27 &$\pm$&  0.25 &    7.36 &$\pm$&  0.69 &   5.95 &$\pm$& 0.52 &  5.63 &$\pm$& 0.45 & 4758.04 &$\pm$& 407.56 &  495.21 &$\pm$& 12.65 \\
      \;\;\emph{sel-init (s)} &  -    - &     &        &  --\;\; &     &       &  --\;\; &     &       & --\;\; &     &      &--\;\; &     &      &  --\;\; &     &        &  489.22 &$\pm$& 12.64 \\
      \;\;\emph{online (s)}   &  166.19 &$\pm$&   9.29 &\bf 3.27 &$\pm$&  0.25 &    7.36 &$\pm$&  0.69 &   5.95 &$\pm$& 0.52 &  5.63 &$\pm$& 0.45 & 4758.04 &$\pm$& 407.56 &    5.99 &$\pm$&  0.48 \\
      \;\;\emph{search (s)}   &    0.01 &$\pm$&   0.00 &    0.01 &$\pm$&  0.00 &    0.02 &$\pm$&  0.00 &   0.01 &$\pm$& 0.00 &  0.01 &$\pm$& 0.00 &    0.02 &$\pm$&   0.00 &    0.03 &$\pm$&  0.00 \\
      \;\;\emph{sel (s)}      &    0.00 &$\pm$&   0.00 &    0.00 &$\pm$&  0.00 &    0.00 &$\pm$&  0.00 &   0.00 &$\pm$& 0.00 &  0.00 &$\pm$& 0.00 & 4750.16 &$\pm$& 406.98 &    2.03 &$\pm$&  0.22 \\
      \;\;\emph{eval (s)}     &  166.17 &$\pm$&   9.28 &\bf 3.26 &$\pm$&  0.25 &    7.34 &$\pm$&  0.69 &   5.93 &$\pm$& 0.52 &  5.61 &$\pm$& 0.45 &    7.82 &$\pm$&   0.61 &    3.91 &$\pm$&  0.27 \\
      \;\;\emph{eval (edges)} &  657.02 &$\pm$&  29.24 &\bf62.24 &$\pm$&  6.12 &   98.54 &$\pm$& 10.89 &  69.96 &$\pm$& 6.98 & 75.88 &$\pm$& 7.47 &\bf66.24 &$\pm$&   6.36 &\bf62.16 &$\pm$&  6.10 \\
      \addlinespace[0.25em]
      ArmPlan3 & & & & & & & & & & & & & & & & & & & & & \\
      \;\;\emph{total (s)}    &  534.16 &$\pm$&  55.64 &    9.61 &$\pm$&  1.33 &\bf 2.50 &$\pm$&  0.23 &   4.91 &$\pm$& 0.56 &  8.47 &$\pm$& 0.99 & 2073.23 &$\pm$& 198.75 &  497.76 &$\pm$& 10.27 \\
      \;\;\emph{sel-init (s)} &  --\;\; &     &        &  --\;\; &     &       &  --\;\; &     &       & --\;\; &     &      &--\;\; &     &      &  --\;\; &     &        &  493.59 &$\pm$& 10.21 \\
      \;\;\emph{online (s)}   &  534.16 &$\pm$&  55.64 &    9.61 &$\pm$&  1.33 &\bf 2.50 &$\pm$&  0.23 &   4.91 &$\pm$& 0.56 &  8.47 &$\pm$& 0.99 & 2073.23 &$\pm$& 198.75 &    4.17 &$\pm$&  0.43 \\
      \;\;\emph{search (s)}   &    0.02 &$\pm$&   0.01 &    0.02 &$\pm$&  0.00 &    0.02 &$\pm$&  0.00 &   0.02 &$\pm$& 0.00 &  0.02 &$\pm$& 0.00 &    0.03 &$\pm$&   0.01 &    0.04 &$\pm$&  0.00 \\
      \;\;\emph{sel (s)}      &    0.00 &$\pm$&   0.00 &    0.00 &$\pm$&  0.00 &    0.00 &$\pm$&  0.00 &   0.00 &$\pm$& 0.00 &  0.00 &$\pm$& 0.00 & 2069.29 &$\pm$& 198.53 &    0.98 &$\pm$&  0.13 \\
      \;\;\emph{eval (s)}     &  534.10 &$\pm$&  55.63 &    9.58 &$\pm$&  1.33 &    2.48 &$\pm$&  0.23 &   4.89 &$\pm$& 0.56 &  8.44 &$\pm$& 0.99 &    3.90 &$\pm$&   0.31 &    3.15 &$\pm$&  0.32 \\
      \;\;\emph{eval (edges)} & 1845.38 &$\pm$& 195.57 &   78.90 &$\pm$& 10.36 &\bf30.70 &$\pm$&  3.62 &  37.04 &$\pm$& 4.59 & 69.26 &$\pm$& 7.97 &\bf30.82 &$\pm$&   3.60 &\bf31.38 &$\pm$&  3.80 \\
      \addlinespace[0.25em]
      \bottomrule
   \end{tabular}%
   }%
   \end{widepage}
   \vspace{0.5cm}
   \caption{
      Detailed timing results for each selector.
      The actual edge weights for the illustrative
      PartConn and UnitSquare problems were pre-computed,
      and therefore their timings are not included.
      The Partition selector requires initialization of the $Z$-values
      (\ref{eqn:partitionfn}) for the graph using only the estimated
      edge weights.
      Since this is not particular to either the actual edge weights
      (e.g. from the obstacle distribution)
      or the start/goal vertices from a particular instance,
      this initialization (\emph{sel-init}) is considered separately.
      The online running time (\emph{online}) is broken into LazySP's
      three primary steps: the inner search (\emph{search}),
      invoking the edge selector (\emph{sel}),
      and evaluating edges (\emph{eval}).
      We also show the number of edges evaluated.}
   \label{fig:table-timing-results}
\end{figure*}
