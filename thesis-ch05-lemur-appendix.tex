\chapter{Appendix: LEMUR Results}
\label{chap:appendix-utility}

\begin{figure*}
   \begin{widepage}
   \begin{center}
    
   \includegraphics{build/lemur-sq/herb-master}

   \caption[Experimental results across six single-query motion planning
      instances for a 7-DOF robot arm.
      Top: expected planning cost $p$ (in seconds)
      vs. execution cost $x$ (in radians) for each parameterized
      planner for various values of their parameters.
      Bottom: the mean negative utility
      $-U = \lambda_U p + (1\!-\!\lambda_U) x$
      (solid lines) measured for each planner
      (lower on plot is better) as the
      utility parameter $\lambda_U$ is varied.
      The 95\% confidence interval of the mean is also shown.
      Planners used the same parameter schedule across the problems
      as shown in Figure~\ref{fig:herbarm-schedules}.
      The per-problem maximum achievable mean utilities
      (\ref{eqn:oracle-param-schedule})
      for each planner are shown as dotted lines.
      Planners are RRT-Connect with shortcutting, BIT*,
      LEMUR (no roadmap cache), and LEMUR (with roadmap cache).
   ][-30cm]{x}
   \label{fig:lemur:sq-herb-master}

   \end{center}
   \end{widepage}

   \vspace{0.1in}
   \smallskip\noindent\small Figure \ref{fig:lemur:sq-herb-master}:
      Experimental results across six single-query motion planning
      instances for a 7-DOF robot arm.
      Top: expected planning cost $p$ (in seconds)
      vs. execution cost $x$ (in radians) for each parameterized
      planner for various values of their parameters.
      Bottom: the mean negative utility
      $-U = \lambda_U p + (1\!-\!\lambda_U) x$
      (solid lines) measured for each planner
      (lower on plot is better) as the
      utility parameter $\lambda_U$ is varied.
      The 95\% confidence interval of the mean is also shown.
      Planners used the same parameter schedule across the problems
      as shown in Figure~\ref{fig:herbarm-schedules}.
      The per-problem maximum achievable mean utilities
      (\ref{eqn:oracle-param-schedule})
      for each planner are shown as dotted lines.
      Planners are
      \protect\tikz{\protect\node[fill=red,draw=black]{};}\;RRT-Connect with shortcutting,
      \protect\tikz{\protect\node[fill=green,draw=black]{};}\;BIT*,
      \protect\tikz{\protect\node[fill=blue,draw=black]{};}\;LEMUR (no roadmap cache),
      and \protect\tikz{\protect\node[fill=black!80,draw=black]{};}\;LEMUR (with roadmap cache).
   
\end{figure*}

\begin{figure*}
   \begin{widepage}
   \begin{center}
   
   \includegraphics{build/lemur-sq/herb-timing}

   \caption[Time breakdown of LEMUR for each of the six
      experimental problems across various values of $\lambda_p$.
      Cumulative time is shown performing the following operations:
      roadmap generation, graph search, collision checking, and unaccounted for.
      The bottom row shows the results of the algorithm over the same
      set of 50 roadmaps when reading the cached roadmap from disk
      to avoid online nearest-neighbor queries.
   ][-30cm]{x}
   \label{fig:lemur:sq-herb-timing}

   \end{center}
   \end{widepage}

   \vspace{0.1in}
   \smallskip\noindent\small Figure \ref{fig:lemur:sq-herb-timing}:
      Time breakdown of LEMUR for each of the six
      experimental problems across various values of $\lambda_p$.
      Cumulative time is shown performing the following operations:
      \protect\tikz{\protect\node[fill=red!20,draw=black]{};}\;roadmap generation,
      \protect\tikz{\protect\node[fill=blue!20,draw=black]{};}\;graph search,
      \protect\tikz{\protect\node[fill=green!20,draw=black]{};}\;collision checking,
      and \protect\tikz{\protect\node[fill=black!20,draw=black]{};}\;unaccounted for.
      The bottom row shows the results of the algorithm over the same
      set of 50 roadmaps when reading the cached roadmap from disk
      to avoid online nearest-neighbor queries.
   
\end{figure*}

\begin{figure*}
   \begin{widepage}
   \centering   
   \includegraphics{build/lemur-sq/workcell-master}
   \end{widepage}
   \caption{Summary of $p$-vs-$x$ and utility results for each step in the
      workcell problem.}
   \label{fig:lemur:sq-workcell-master}
\end{figure*}

\begin{figure*}
   \begin{widepage}
   \centering   
   \includegraphics{build/lemur-sq/workcell-timing}
   \end{widepage}
   \caption{Timing results for each step in the workcell problem.}
   \label{fig:lemur:sq-workcell-timing}
\end{figure*}
