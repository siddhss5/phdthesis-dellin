\chapter{Sub-Problems in Manipulation Planning}
\label{chap:formulation}

In this chapter,
we lay out the structure of the manipulation planning problem.

Definitely reference DRC Trials paper for planning vs. execution
time breakdown!

\section{Sub-Problem Structure}

{
\setlength{\offsetpage}{0.5in}
\begin{figure}
\begin{widepage}
\begin{center}

\begin{subfigure}[t]{0.19\linewidth}
\centering
\includegraphics[width=\columnwidth]{figs/testherb-a.png}
\caption{Start config}
\end{subfigure}
\begin{subfigure}[t]{0.19\linewidth}
\centering
\includegraphics[width=\columnwidth]{figs/testherb-b.png}
\caption{Step 1 in $X_1$}
\end{subfigure}
\begin{subfigure}[t]{0.19\linewidth}
\centering
\includegraphics[width=\columnwidth]{figs/testherb-c.png}
\caption{Step 2 in $X_2$}
\end{subfigure}
\begin{subfigure}[t]{0.19\linewidth}
\centering
\includegraphics[width=\columnwidth]{figs/testherb-d.png}
\caption{Step 3 in $X_3$}
\end{subfigure}
\begin{subfigure}[t]{0.19\linewidth}
\centering
\includegraphics[width=\columnwidth]{figs/testherb-e.png}
\caption{End config}
\end{subfigure}

\vspace{0.3cm}

\begin{subfigure}[t]{\linewidth}
\begin{center}
\begin{tikzpicture}
\tikzset{>=latex} % arrow heads

% top: symbolic planner
\node[draw,ellipse,align=center] (high) at (0,0)
   {Task Planner:\\Clear the table};
\node[draw,circle,align=center,inner sep=0,minimum size=1.9cm]
   (step1) at (-4.2,-2)
   {Step 1:\\Transit to\\[-0.04in]grasp};
\node (grasp) at (-2.1,-2) {Grasp};
\node[draw,circle,align=center,inner sep=0,minimum size=1.9cm]
   (step2) at (0,-2)
   {Step 2:\\Transfer to\\[-0.04in]drop};
\node (drop) at (2.1,-2) {Drop};
\node[draw,circle,align=center,inner sep=0,minimum size=1.9cm]
   (step3) at (4.2,-2)
   {Step 3:\\Transit to\\[-0.04in]home};
\draw[->] (high) -- (step1);
\draw[->] (high) -- (step2);
\draw[->] (high) -- (step3);
\draw[->] (step1) -- (grasp);
\draw[->] (grasp) -- (step2);
\draw[->] (step2) -- (drop);
\draw[->] (drop) -- (step3);

% middle: c-space sequence
% valid subsets
\node[draw,black,rounded corners,
   minimum height=2.3cm,minimum width=4.5cm]
   (X1) at (-3.5,-4.75) {};
\node[draw,black,rounded corners,
   minimum height=2.6cm,minimum width=5.5cm]
   (X2) at (0,-4.75) {};
\node[draw,black,rounded corners,
   minimum height=2.3cm,minimum width=5.5cm]
   (X3) at ( 4,-4.75) {};
% root sets
\node[draw,black,rounded corners,
   minimum height=2cm,minimum width=1.2cm]
   (Xgrasp) at (-2,-4.75) {};
\node[draw,black,rounded corners,
   minimum height=2cm,minimum width=1.2cm]
   (Xdrop) at (2,-4.75) {};
\node[draw,black,rounded corners,
   minimum height=2cm,minimum width=1.2cm]
   (Xend) at (6,-4.75) {};
% set labels
\node[above=-0.6cm of Xgrasp] {$X_{\mbox{\scriptsize grasp}}$};
\node[above=-0.6cm of Xdrop] {$X_{\mbox{\scriptsize drop}}$};
\node[above=-0.6cm of Xend] {$X_{\mbox{\scriptsize end}}$};
\node[above left=-0.6cm and -2.1cm of X1] {$X_1$};
\node[above=-0.75cm of X2] {$X_2$};
\node[above=-0.6cm of X3] {$X_3$};
% nodes and paths
\node[circle,fill=black,inner sep=2] (xstart) at (-5,-4.9) {};
\node[circle,fill=black!50,inner sep=2] (xg1) at (-1.9,-4.6) {};
\node[circle,fill=black,inner sep=2] (xg2) at (-2.1,-5.3) {};
\node[circle,fill=black,inner sep=2] (xd1) at ( 1.9,-4.5) {};
\node[circle,fill=black!50,inner sep=2] (xd2) at ( 2.1,-5.0) {};
\node[circle,fill=black!50,inner sep=2] (xd3) at ( 2.0,-5.4) {};
\node[circle,fill=black!50,inner sep=2] (xe1) at ( 6.1,-4.7) {};
\node[circle,fill=black,inner sep=2] (xe2) at ( 5.9,-5.2) {};
\node[above=0cm of xstart] {$x_{\mbox{\scriptsize start}}$};
% lines
\draw[line width=1.5mm,white]
   (xstart) -- (xg1) (xg1) -- (xd1) (xg1) -- (xd2) (xg1) -- (xd3)
   (xg2) -- (xd2) (xg2) -- (xd3) (xd1) -- (xe1) (xd2) -- (xe1)
   (xd2) -- (xe2) (xd3) -- (xe1) (xd3) -- (xe2);
\draw[draw=black!50]
   (xstart) -- (xg1) (xg1) -- (xd1) (xg1) -- (xd2) (xg1) -- (xd3)
   (xg2) -- (xd2) (xg2) -- (xd3) (xd1) -- (xe1) (xd2) -- (xe1)
   (xd2) -- (xe2) (xd3) -- (xe1) (xd3) -- (xe2);
\draw[line width=1.5mm,white]
   (xstart) -- (xg2) (xg2) -- (xd1) (xd1) -- (xe2);
\draw
   (xstart) -- (xg2) (xg2) -- (xd1) (xd1) -- (xe2);
% grey sets (overlay)
\node[fill=black,opacity=0.1,rounded corners,
   minimum height=2cm,minimum width=1.2cm]
   at (-2,-4.75) {};
\node[fill=black,opacity=0.1,rounded corners,
   minimum height=2cm,minimum width=1.2cm]
   at (2,-4.75) {};
\node[fill=black,opacity=0.1,rounded corners,
   minimum height=2cm,minimum width=1.2cm]
   at (6,-4.75) {};
% question mark bubbles
\node[circle,fill=white,fill opacity=0.9,inner sep=7pt]
   at (-3.75,-4.9) {\LARGE ?};
\node[circle,fill=white,fill opacity=0.9,inner sep=7pt]
   at (0,-4.9) {\LARGE ?};
\node[circle,fill=white,fill opacity=0.9,inner sep=7pt]
   at (4.0,-4.9) {\LARGE ?};

% bottom: individual planner instances
\node[draw,black,rounded corners,
   minimum height=2cm,minimum width=3.5cm]
   (sm1) at (-4.2,-7.75) {};
\node[draw,black,fill=black!10,rounded corners,
   minimum height=1.7cm,minimum width=0.8cm,
   right=-1cm of sm1] {};
\node[circle,fill=black!50,inner sep=2] (sm1xstart) at (-5.4,-7.75) {};
\node[circle,fill=black!50,inner sep=2] (sm1xg1) at (-2.9,-7.5) {};
\node[circle,fill=black!50,inner sep=2] (sm1xg2) at (-3.1,-8.1) {};
\draw[draw=black!50]
   (sm1xstart) -- (sm1xg1) (sm1xstart) -- (sm1xg2);
\node[above=-0.5cm of sm1] {\footnotesize $X_1$};
\node[circle,fill=white,fill opacity=0.9,inner sep=5pt]
   at (-4.2,-7.85) {\Large ?};

\node[draw,black,rounded corners,
   minimum height=2cm,minimum width=3.5cm]
   (sm2) at (0.5,-7.75) {};
\node[draw,black,fill=black!10,rounded corners,
   minimum height=1.7cm,minimum width=0.8cm,
   left=-1cm of sm2] {};
\node[draw,black,fill=black!10,rounded corners,
   minimum height=1.7cm,minimum width=0.8cm,
   right=-1cm of sm2] {};
\node[circle,fill=black!50,inner sep=2] (sm2xg1) at (-0.6,-7.5) {};
\node[circle,fill=black!50,inner sep=2] (sm2xg2) at (-0.8,-8.1) {};
\node[circle,fill=black!50,inner sep=2] (sm2xd1) at ( 1.6,-7.4) {};
\node[circle,fill=black!50,inner sep=2] (sm2xd2) at ( 1.8,-7.9) {};
\node[circle,fill=black!50,inner sep=2] (sm2xd3) at ( 1.7,-8.3) {};
\draw[draw=black!50]
   (sm2xg1) -- (sm2xd1) (sm2xg1) -- (sm2xd2) (sm2xg1) -- (sm2xd3)
   (sm2xg2) -- (sm2xd1) (sm2xg2) -- (sm2xd2) (sm2xg2) -- (sm2xd3);
\node[above=-0.5cm of sm2] {\footnotesize $X_2$};
\node[circle,fill=white,fill opacity=0.9,inner sep=5pt]
   at (0.5,-7.85) {\Large ?};

\node[draw,black,rounded corners,
   minimum height=2cm,minimum width=3.5cm]
   (sm3) at (5.2,-7.75) {};
\node[draw,black,fill=black!10,rounded corners,
   minimum height=1.7cm,minimum width=0.8cm,
   left=-1cm of sm3] {};
\node[draw,black,fill=black!10,rounded corners,
   minimum height=1.7cm,minimum width=0.8cm,
   right=-1cm of sm3] {};
\node[circle,fill=black!50,inner sep=2] (sm3xd1) at ( 4.0,-7.4) {};
\node[circle,fill=black!50,inner sep=2] (sm3xd2) at ( 4.2,-7.9) {};
\node[circle,fill=black!50,inner sep=2] (sm3xd3) at ( 4.1,-8.3) {};
\node[circle,fill=black!50,inner sep=2] (sm3xe1) at ( 6.4,-7.6) {};
\node[circle,fill=black!50,inner sep=2] (sm3xe2) at ( 6.2,-8.1) {};
\draw[draw=black!50]
   (sm3xd1) -- (sm3xe1) (sm3xd1) -- (sm3xe2) (sm3xd2) -- (sm3xe1)
   (sm3xd2) -- (sm3xe2) (sm3xd3) -- (sm3xe1) (sm3xd3) -- (sm3xe2);
\node[above=-0.5cm of sm3] {\footnotesize $X_3$};
\node[circle,fill=white,fill opacity=0.9,inner sep=5pt]
   at (5.2,-7.85) {\Large ?};

% arrows
\draw[->] (-4.1,-6.1) -- (-4.4,-6.6);
\draw[->] ( 0.3,-6.2) -- ( 0.4,-6.6);
\draw[->] ( 4.5,-6.1) -- ( 4.8,-6.6);

% side labels
\node[draw,rotate=90,align=center] at (-6.7,-2)
   {Symbolic\\Plan};
\node[draw,rotate=90,align=center] at (-6.7,-4.75)
   {$\mathcal{C}$-space\\Planning};
\node[draw,rotate=90,align=center] at (-6.7,-7.75)
   {Individual\\Planner Queries};

\end{tikzpicture}
\end{center}
\caption{This thesis focuses on efficient geometric planning
   for manipulation tasks (the lower two levels here).
   Task planning can be performed by an autonomous
   symbolic planner,
   or guided by a human operator.}
\end{subfigure}

\end{center}
\caption{Diagram of multi-step planning framework.}
\label{fig:diagram-multi-step}
\end{widepage}
\end{figure}
}

The manipulation problem has a particular structure,
which we discuss here.
See Figure~\ref{fig:diagram-multi-step}
for a diagram.

Briefly discuss the higher-level planner.
Not doing research here -- just need something that specifies
geometric goals.
Complementary to symbolic planners
or from humans (e.g. the DRC -- need cite).
For this proposal,
will only discuss seqential sub-plans,
but an intelligent meta-planner
can do non-sequential stuff too.

\section{Motivation: DRC Trials}

Talk a lot about the DRC approach.
It was slow, and it got stuck!

Copy-paste in from the ISER paper!

\section{Approach}

Multi-step problem structure (lots of options at each step);
decomposition into a bunch of "local planner" like things.
We're essentially building a meta-graph.
Don't get stuck.

We talk a lot about the mapping
from continuous spaces to graphs in
Chapter~\ref{chap:graphs-in-continuous}.

Talk about subgoals in A* literature.

For now, assume a prescribed order,
but we'll talk later about more complex meta-planning
approaches.

\subsection{Root Sampling}

Do sampling at this higher level.
(Can't rely on the sub-planners to generate their own starts/goals --
they must be synchronized.)
Potential research question: learn good intermediate goals.

\subsection{Two Ways to Make it Fast}

The proposal is split into two complementary things,
guided by how slow collision checking is.

Chapters 3,4 focus WITHIN each step.
Make each step fast.

Chapters 5,6 focus BETWEEN steps.
Exploit structure between steps (Chapter~\ref{chap:multi-set}).
Single-query vs. multi-query.

\section{Related Work}

While we will primarily apply graph search techniques to this problem,
we start by reviewing alternative approaches.

\section{Review of Alternative Approaches}
\label{sec:related-work}

Since $C$ is continuous,
all approaches must introduce some sort of discretization
in order to compute solutions.
We choose to build a graph consisting of vertices and edges in $C$,
and then search that graph (Section~\ref{sec:best-first}).
We do this because we can rely on existing techniques,
and because an explicit graph can be more easily reused than other
approaches.
In this section, we discuss alternative approaches to solving
the motion planning problem for articulated robots.

%\subsection{Multi-Query Approaches}
%
%We could run a PRM \cite{kavrakietal1996prm}.
%Commit to a fixed graph,
%and determine the validity of each vertex and edge w.r.t.
%$\mathcal{C}_{\mbox{\scriptsize free}}$.
%Then, at query time,
%run A* to find the shortest path (this is fast due to graph sparseness).
%This is good because it reuses work.
%Unfortunately,
%(a) our $\mathcal{C}_{\mbox{\scriptsize free}}$
%is different for every subplan
%(and for different options with each),
%and (b) we don't want to determine validity over the entire graph
%because it's costly.

\subsection{Anytime algorithms}

Compare to RRT*, FMT*, BIT*, etc.

\subsection{Other}

Need to look into the SBL planner \cite{sanchezante2001sbl}
(Single Query BiDirectional Lazy PRM).

\subsection{Incremental Construction Algorithms}

We could construct the graph incrementally and in response to the shape
of $\mathcal{C}_{\mbox{\scriptsize free}}$.
RRTs behave well for quickly finding feasible paths.
We'll compare against them at the end of this chapter.
Also talk about ESTs.
Difficult to cache things.

\subsection{Trajectory Optimization}

One approach is trajectory optimization.
For example, there's CHOMP \cite{zucker2013chomp}
and TrajOpt \cite{schulman2013trajopt}.
Lots of other prior work here that is not manipulation-focused.

Local minima problems.
Difficult to cache / apply to similar problems.

This is largely complementary.
Use sampling-based planning to quickly find feasible solutions,
and then optimize them.

\section{Other Stuff}

Other stuff to touch on:
\begin{itemize}
\item Dealing with constraints
\item Dealing with dual-arm stuff
\item Dealing with optimizers (run afterwards!)
   Most solution paths will be unexecuted, so optimize later!
\end{itemize}
