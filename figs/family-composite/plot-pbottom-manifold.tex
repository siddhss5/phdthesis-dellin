\documentclass{standalone}
\usepackage{pgfplots}
\usepgfplotslibrary{patchplots}
\begin{document}
\begin{tikzpicture}

\begin{axis}[
   width=6.5cm,
   height=8cm,
   %xlabel=x,
   %ylabel=y,
   xmin=-0.05, xmax=1.05,
   ymin=-0.05, ymax=1.05,
   zmin=-1.5, zmax=1,
   xtick=\empty, ytick=\empty, ztick=\empty,
   view={-30}{25},
   %view={0}{0},
   axis lines=none,
   ]

   % projected surface
   \draw[black]
      (axis cs:0,0,-1.5) -- (axis cs:0,1,-1.5)
      -- (axis cs:1,1,-1.5) -- (axis cs:1,0,-1.5) -- cycle;
      
   % background projection lines (BEHIND)
   \draw[black!50,dotted]
      (axis cs:0,1,-1.5) -- (axis cs:0,1,0.0)
      (axis cs:1,0,-1.5) -- (axis cs:1,0,0.0)
      (axis cs:1,1,-1.5) -- (axis cs:1,1,0.0);
   
   \addplot3[thick,fill=green,opacity=0.5] coordinates {
      (0.8,1.0,-1.5)
      (1.0,1.0,-1.5)
      (1.0,0.5,-1.5)
      (0.8,0.5,-1.5)
      (0.8,1.0,-1.5)
   };
   
   \addplot3[thick,fill=yellow,opacity=0.5] table[x=x, y=y, z expr=-1.5]
      {figs/family-composite/pbottom-ro-intersection.txt};
   
   % draw path
   \addplot3[thick,smooth] table[x=x, y=y, z expr=-1.5] {figs/family-composite/path-transit2.txt};
   \node[inner sep=1pt,fill=black,circle] at (axis cs:0.42,0.37,-1.5) {}; % release
   \node[inner sep=1pt,fill=black,circle] at (axis cs:0.90,0.10,-1.5) {}; % end
   
   % background projection lines (IN FRONT)
   \draw[black!50,dotted]
      (axis cs:0,0,-1.5) -- (axis cs:0,0,0.0);
   
   % draw composite space bounds (BEHIND)
   \draw[black!50]
      (axis cs:0,0,0) -- (axis cs:0,1,0)
         -- (axis cs:1,1,0) -- (axis cs:1,0,0) -- cycle % bottom
      (axis cs:0,1,0) -- (axis cs:0,1,1)
      (axis cs:1,0,0) -- (axis cs:1,0,1)
      (axis cs:1,1,0) -- (axis cs:1,1,1) % sides
      (axis cs:0,1,1) -- (axis cs:1,1,1) -- (axis cs:1,0,1); % top

%   \fill[blue!50,opacity=0.5,even odd rule]
(axis cs:0,0,0.25) -- (axis cs:1,0,0.25) -- (axis cs:1,1,0.25) -- (axis cs:0,1,0.25) -- cycle
(axis cs:0.0,0.655,0.25) -- (axis cs:0.01,0.655,0.25) -- (axis cs:0.02,0.655,0.25) -- (axis cs:0.03,0.655,0.25) -- (axis cs:0.04,0.655,0.25) -- (axis cs:0.05,0.655,0.25) -- (axis cs:0.06,0.655,0.25) -- (axis cs:0.07,0.655,0.25) -- (axis cs:0.08,0.655,0.25) -- (axis cs:0.09,0.655,0.25) -- (axis cs:0.1,0.655,0.25) -- (axis cs:0.11,0.655,0.25) -- (axis cs:0.12,0.655,0.25) -- (axis cs:0.125,0.65,0.25) -- (axis cs:0.125,0.64,0.25) -- (axis cs:0.125,0.63,0.25) -- (axis cs:0.12,0.625,0.25) -- (axis cs:0.115,0.62,0.25) -- (axis cs:0.115,0.61,0.25) -- (axis cs:0.115,0.6,0.25) -- (axis cs:0.115,0.5900000000000001,0.25) -- (axis cs:0.115,0.5800000000000001,0.25) -- (axis cs:0.11,0.575,0.25) -- (axis cs:0.105,0.5700000000000001,0.25) -- (axis cs:0.105,0.56,0.25) -- (axis cs:0.105,0.55,0.25) -- (axis cs:0.105,0.54,0.25) -- (axis cs:0.105,0.53,0.25) -- (axis cs:0.105,0.52,0.25) -- (axis cs:0.105,0.51,0.25) -- (axis cs:0.105,0.5,0.25) -- (axis cs:0.1,0.495,0.25) -- (axis cs:0.095,0.49,0.25) -- (axis cs:0.095,0.48,0.25) -- (axis cs:0.095,0.47,0.25) -- (axis cs:0.095,0.45999999999999996,0.25) -- (axis cs:0.095,0.44999999999999996,0.25) -- (axis cs:0.095,0.43999999999999995,0.25) -- (axis cs:0.095,0.43000000000000005,0.25) -- (axis cs:0.095,0.42000000000000004,0.25) -- (axis cs:0.095,0.41000000000000003,0.25) -- (axis cs:0.095,0.4,0.25) -- (axis cs:0.095,0.39,0.25) -- (axis cs:0.095,0.38,0.25) -- (axis cs:0.09,0.375,0.25) -- (axis cs:0.085,0.37,0.25) -- (axis cs:0.085,0.36,0.25) -- (axis cs:0.085,0.35,0.25) -- (axis cs:0.09,0.345,0.25) -- (axis cs:0.1,0.345,0.25) -- (axis cs:0.11,0.345,0.25) -- (axis cs:0.12,0.345,0.25) -- (axis cs:0.13,0.345,0.25) -- (axis cs:0.14,0.345,0.25) -- (axis cs:0.15,0.345,0.25) -- (axis cs:0.16,0.345,0.25) -- (axis cs:0.17,0.345,0.25) -- (axis cs:0.18,0.345,0.25) -- (axis cs:0.19,0.345,0.25) -- (axis cs:0.2,0.345,0.25) -- (axis cs:0.21,0.345,0.25) -- (axis cs:0.22,0.345,0.25) -- (axis cs:0.23,0.345,0.25) -- (axis cs:0.24,0.345,0.25) -- (axis cs:0.25,0.345,0.25) -- (axis cs:0.26,0.345,0.25) -- (axis cs:0.27,0.345,0.25) -- (axis cs:0.28,0.345,0.25) -- (axis cs:0.29,0.345,0.25) -- (axis cs:0.3,0.345,0.25) -- (axis cs:0.31,0.345,0.25) -- (axis cs:0.32,0.345,0.25) -- (axis cs:0.33,0.345,0.25) -- (axis cs:0.34,0.345,0.25) -- (axis cs:0.35,0.345,0.25) -- (axis cs:0.36,0.345,0.25) -- (axis cs:0.37,0.345,0.25) -- (axis cs:0.38,0.345,0.25) -- (axis cs:0.39,0.345,0.25) -- (axis cs:0.4,0.345,0.25) -- (axis cs:0.41,0.345,0.25) -- (axis cs:0.42,0.345,0.25) -- (axis cs:0.43,0.345,0.25) -- (axis cs:0.44,0.345,0.25) -- (axis cs:0.45,0.345,0.25) -- (axis cs:0.46,0.345,0.25) -- (axis cs:0.47,0.345,0.25) -- (axis cs:0.48,0.345,0.25) -- (axis cs:0.49,0.345,0.25) -- (axis cs:0.5,0.345,0.25) -- (axis cs:0.51,0.345,0.25) -- (axis cs:0.52,0.345,0.25) -- (axis cs:0.53,0.345,0.25) -- (axis cs:0.54,0.345,0.25) -- (axis cs:0.545,0.35,0.25) -- (axis cs:0.545,0.36,0.25) -- (axis cs:0.55,0.365,0.25) -- (axis cs:0.555,0.37,0.25) -- (axis cs:0.555,0.38,0.25) -- (axis cs:0.56,0.385,0.25) -- (axis cs:0.565,0.39,0.25) -- (axis cs:0.565,0.4,0.25) -- (axis cs:0.57,0.405,0.25) -- (axis cs:0.575,0.41000000000000003,0.25) -- (axis cs:0.58,0.41500000000000004,0.25) -- (axis cs:0.585,0.42000000000000004,0.25) -- (axis cs:0.59,0.42500000000000004,0.25) -- (axis cs:0.595,0.43000000000000005,0.25) -- (axis cs:0.6,0.43500000000000005,0.25) -- (axis cs:0.61,0.43500000000000005,0.25) -- (axis cs:0.62,0.43500000000000005,0.25) -- (axis cs:0.63,0.43500000000000005,0.25) -- (axis cs:0.64,0.43500000000000005,0.25) -- (axis cs:0.645,0.43999999999999995,0.25) -- (axis cs:0.65,0.44499999999999995,0.25) -- (axis cs:0.66,0.44499999999999995,0.25) -- (axis cs:0.67,0.44499999999999995,0.25) -- (axis cs:0.675,0.43999999999999995,0.25) -- (axis cs:0.68,0.43500000000000005,0.25) -- (axis cs:0.685,0.43000000000000005,0.25) -- (axis cs:0.69,0.42500000000000004,0.25) -- (axis cs:0.695,0.42000000000000004,0.25) -- (axis cs:0.7,0.41500000000000004,0.25) -- (axis cs:0.705,0.41000000000000003,0.25) -- (axis cs:0.71,0.405,0.25) -- (axis cs:0.715,0.4,0.25) -- (axis cs:0.72,0.395,0.25) -- (axis cs:0.725,0.39,0.25) -- (axis cs:0.725,0.38,0.25) -- (axis cs:0.73,0.375,0.25) -- (axis cs:0.735,0.37,0.25) -- (axis cs:0.735,0.36,0.25) -- (axis cs:0.74,0.355,0.25) -- (axis cs:0.745,0.35,0.25) -- (axis cs:0.75,0.345,0.25) -- (axis cs:0.76,0.345,0.25) -- (axis cs:0.77,0.345,0.25) -- (axis cs:0.78,0.345,0.25) -- (axis cs:0.79,0.345,0.25) -- (axis cs:0.8,0.345,0.25) -- (axis cs:0.81,0.345,0.25) -- (axis cs:0.82,0.345,0.25) -- (axis cs:0.83,0.345,0.25) -- (axis cs:0.84,0.345,0.25) -- (axis cs:0.85,0.345,0.25) -- (axis cs:0.86,0.345,0.25) -- (axis cs:0.87,0.345,0.25) -- (axis cs:0.88,0.345,0.25) -- (axis cs:0.89,0.345,0.25) -- (axis cs:0.9,0.345,0.25) -- (axis cs:0.91,0.345,0.25) -- (axis cs:0.92,0.345,0.25) -- (axis cs:0.93,0.345,0.25) -- (axis cs:0.94,0.345,0.25) -- (axis cs:0.95,0.345,0.25) -- (axis cs:0.96,0.345,0.25) -- (axis cs:0.97,0.345,0.25) -- (axis cs:0.98,0.345,0.25) -- (axis cs:0.99,0.345,0.25) -- (axis cs:1.0,0.345,0.25) -- (axis cs:1.0,0.0,0.25) -- (axis cs:0.0,0.0,0.25) -- cycle
;


   %%% first, draw stuff inside the green box
   
   % part of eo top surface BEHIND (or INSIDE OF) RxO
   \begin{scope}
      \clip (axis cs:0.8,1.0,0.25) -- (axis cs:1.0,1.0,0.25) -- (axis cs:1.0,0.5,0.25) -- (axis cs:0.8,0.5,0.25) -- cycle;
      \fill[blue!50,opacity=0.5,even odd rule]
(axis cs:0,0,0.25) -- (axis cs:1,0,0.25) -- (axis cs:1,1,0.25) -- (axis cs:0,1,0.25) -- cycle
(axis cs:0.0,0.655,0.25) -- (axis cs:0.01,0.655,0.25) -- (axis cs:0.02,0.655,0.25) -- (axis cs:0.03,0.655,0.25) -- (axis cs:0.04,0.655,0.25) -- (axis cs:0.05,0.655,0.25) -- (axis cs:0.06,0.655,0.25) -- (axis cs:0.07,0.655,0.25) -- (axis cs:0.08,0.655,0.25) -- (axis cs:0.09,0.655,0.25) -- (axis cs:0.1,0.655,0.25) -- (axis cs:0.11,0.655,0.25) -- (axis cs:0.12,0.655,0.25) -- (axis cs:0.125,0.65,0.25) -- (axis cs:0.125,0.64,0.25) -- (axis cs:0.125,0.63,0.25) -- (axis cs:0.12,0.625,0.25) -- (axis cs:0.115,0.62,0.25) -- (axis cs:0.115,0.61,0.25) -- (axis cs:0.115,0.6,0.25) -- (axis cs:0.115,0.5900000000000001,0.25) -- (axis cs:0.115,0.5800000000000001,0.25) -- (axis cs:0.11,0.575,0.25) -- (axis cs:0.105,0.5700000000000001,0.25) -- (axis cs:0.105,0.56,0.25) -- (axis cs:0.105,0.55,0.25) -- (axis cs:0.105,0.54,0.25) -- (axis cs:0.105,0.53,0.25) -- (axis cs:0.105,0.52,0.25) -- (axis cs:0.105,0.51,0.25) -- (axis cs:0.105,0.5,0.25) -- (axis cs:0.1,0.495,0.25) -- (axis cs:0.095,0.49,0.25) -- (axis cs:0.095,0.48,0.25) -- (axis cs:0.095,0.47,0.25) -- (axis cs:0.095,0.45999999999999996,0.25) -- (axis cs:0.095,0.44999999999999996,0.25) -- (axis cs:0.095,0.43999999999999995,0.25) -- (axis cs:0.095,0.43000000000000005,0.25) -- (axis cs:0.095,0.42000000000000004,0.25) -- (axis cs:0.095,0.41000000000000003,0.25) -- (axis cs:0.095,0.4,0.25) -- (axis cs:0.095,0.39,0.25) -- (axis cs:0.095,0.38,0.25) -- (axis cs:0.09,0.375,0.25) -- (axis cs:0.085,0.37,0.25) -- (axis cs:0.085,0.36,0.25) -- (axis cs:0.085,0.35,0.25) -- (axis cs:0.09,0.345,0.25) -- (axis cs:0.1,0.345,0.25) -- (axis cs:0.11,0.345,0.25) -- (axis cs:0.12,0.345,0.25) -- (axis cs:0.13,0.345,0.25) -- (axis cs:0.14,0.345,0.25) -- (axis cs:0.15,0.345,0.25) -- (axis cs:0.16,0.345,0.25) -- (axis cs:0.17,0.345,0.25) -- (axis cs:0.18,0.345,0.25) -- (axis cs:0.19,0.345,0.25) -- (axis cs:0.2,0.345,0.25) -- (axis cs:0.21,0.345,0.25) -- (axis cs:0.22,0.345,0.25) -- (axis cs:0.23,0.345,0.25) -- (axis cs:0.24,0.345,0.25) -- (axis cs:0.25,0.345,0.25) -- (axis cs:0.26,0.345,0.25) -- (axis cs:0.27,0.345,0.25) -- (axis cs:0.28,0.345,0.25) -- (axis cs:0.29,0.345,0.25) -- (axis cs:0.3,0.345,0.25) -- (axis cs:0.31,0.345,0.25) -- (axis cs:0.32,0.345,0.25) -- (axis cs:0.33,0.345,0.25) -- (axis cs:0.34,0.345,0.25) -- (axis cs:0.35,0.345,0.25) -- (axis cs:0.36,0.345,0.25) -- (axis cs:0.37,0.345,0.25) -- (axis cs:0.38,0.345,0.25) -- (axis cs:0.39,0.345,0.25) -- (axis cs:0.4,0.345,0.25) -- (axis cs:0.41,0.345,0.25) -- (axis cs:0.42,0.345,0.25) -- (axis cs:0.43,0.345,0.25) -- (axis cs:0.44,0.345,0.25) -- (axis cs:0.45,0.345,0.25) -- (axis cs:0.46,0.345,0.25) -- (axis cs:0.47,0.345,0.25) -- (axis cs:0.48,0.345,0.25) -- (axis cs:0.49,0.345,0.25) -- (axis cs:0.5,0.345,0.25) -- (axis cs:0.51,0.345,0.25) -- (axis cs:0.52,0.345,0.25) -- (axis cs:0.53,0.345,0.25) -- (axis cs:0.54,0.345,0.25) -- (axis cs:0.545,0.35,0.25) -- (axis cs:0.545,0.36,0.25) -- (axis cs:0.55,0.365,0.25) -- (axis cs:0.555,0.37,0.25) -- (axis cs:0.555,0.38,0.25) -- (axis cs:0.56,0.385,0.25) -- (axis cs:0.565,0.39,0.25) -- (axis cs:0.565,0.4,0.25) -- (axis cs:0.57,0.405,0.25) -- (axis cs:0.575,0.41000000000000003,0.25) -- (axis cs:0.58,0.41500000000000004,0.25) -- (axis cs:0.585,0.42000000000000004,0.25) -- (axis cs:0.59,0.42500000000000004,0.25) -- (axis cs:0.595,0.43000000000000005,0.25) -- (axis cs:0.6,0.43500000000000005,0.25) -- (axis cs:0.61,0.43500000000000005,0.25) -- (axis cs:0.62,0.43500000000000005,0.25) -- (axis cs:0.63,0.43500000000000005,0.25) -- (axis cs:0.64,0.43500000000000005,0.25) -- (axis cs:0.645,0.43999999999999995,0.25) -- (axis cs:0.65,0.44499999999999995,0.25) -- (axis cs:0.66,0.44499999999999995,0.25) -- (axis cs:0.67,0.44499999999999995,0.25) -- (axis cs:0.675,0.43999999999999995,0.25) -- (axis cs:0.68,0.43500000000000005,0.25) -- (axis cs:0.685,0.43000000000000005,0.25) -- (axis cs:0.69,0.42500000000000004,0.25) -- (axis cs:0.695,0.42000000000000004,0.25) -- (axis cs:0.7,0.41500000000000004,0.25) -- (axis cs:0.705,0.41000000000000003,0.25) -- (axis cs:0.71,0.405,0.25) -- (axis cs:0.715,0.4,0.25) -- (axis cs:0.72,0.395,0.25) -- (axis cs:0.725,0.39,0.25) -- (axis cs:0.725,0.38,0.25) -- (axis cs:0.73,0.375,0.25) -- (axis cs:0.735,0.37,0.25) -- (axis cs:0.735,0.36,0.25) -- (axis cs:0.74,0.355,0.25) -- (axis cs:0.745,0.35,0.25) -- (axis cs:0.75,0.345,0.25) -- (axis cs:0.76,0.345,0.25) -- (axis cs:0.77,0.345,0.25) -- (axis cs:0.78,0.345,0.25) -- (axis cs:0.79,0.345,0.25) -- (axis cs:0.8,0.345,0.25) -- (axis cs:0.81,0.345,0.25) -- (axis cs:0.82,0.345,0.25) -- (axis cs:0.83,0.345,0.25) -- (axis cs:0.84,0.345,0.25) -- (axis cs:0.85,0.345,0.25) -- (axis cs:0.86,0.345,0.25) -- (axis cs:0.87,0.345,0.25) -- (axis cs:0.88,0.345,0.25) -- (axis cs:0.89,0.345,0.25) -- (axis cs:0.9,0.345,0.25) -- (axis cs:0.91,0.345,0.25) -- (axis cs:0.92,0.345,0.25) -- (axis cs:0.93,0.345,0.25) -- (axis cs:0.94,0.345,0.25) -- (axis cs:0.95,0.345,0.25) -- (axis cs:0.96,0.345,0.25) -- (axis cs:0.97,0.345,0.25) -- (axis cs:0.98,0.345,0.25) -- (axis cs:0.99,0.345,0.25) -- (axis cs:1.0,0.345,0.25) -- (axis cs:1.0,0.0,0.25) -- (axis cs:0.0,0.0,0.25) -- cycle
;

   \end{scope}
   
   % part of pbottom manifold BEHIND (or INSIDE OF) RxO, with boundary
   \begin{scope}
      \clip (axis cs:0.8,1.0,0.35) -- (axis cs:1.0,1.0,0.35) -- (axis cs:1.0,0.5,0.35) -- (axis cs:0.8,0.5,0.35) -- cycle;
      \fill[red!50,opacity=0.5,even odd rule]
(axis cs:0,0,0.35) -- (axis cs:1,0,0.35) -- (axis cs:1,1,0.35) -- (axis cs:0,1,0.35) -- cycle
(axis cs:0.0,0.655,0.35) -- (axis cs:0.01,0.655,0.35) -- (axis cs:0.02,0.655,0.35) -- (axis cs:0.03,0.655,0.35) -- (axis cs:0.04,0.655,0.35) -- (axis cs:0.05,0.655,0.35) -- (axis cs:0.06,0.655,0.35) -- (axis cs:0.07,0.655,0.35) -- (axis cs:0.08,0.655,0.35) -- (axis cs:0.09,0.655,0.35) -- (axis cs:0.1,0.655,0.35) -- (axis cs:0.11,0.655,0.35) -- (axis cs:0.12,0.655,0.35) -- (axis cs:0.13,0.655,0.35) -- (axis cs:0.14,0.655,0.35) -- (axis cs:0.15,0.655,0.35) -- (axis cs:0.16,0.655,0.35) -- (axis cs:0.17,0.655,0.35) -- (axis cs:0.18,0.655,0.35) -- (axis cs:0.19,0.655,0.35) -- (axis cs:0.2,0.655,0.35) -- (axis cs:0.21,0.655,0.35) -- (axis cs:0.22,0.655,0.35) -- (axis cs:0.23,0.655,0.35) -- (axis cs:0.235,0.65,0.35) -- (axis cs:0.23,0.645,0.35) -- (axis cs:0.225,0.64,0.35) -- (axis cs:0.225,0.63,0.35) -- (axis cs:0.22,0.625,0.35) -- (axis cs:0.215,0.62,0.35) -- (axis cs:0.215,0.61,0.35) -- (axis cs:0.21,0.605,0.35) -- (axis cs:0.205,0.6,0.35) -- (axis cs:0.205,0.5900000000000001,0.35) -- (axis cs:0.205,0.5800000000000001,0.35) -- (axis cs:0.2,0.575,0.35) -- (axis cs:0.195,0.5700000000000001,0.35) -- (axis cs:0.195,0.56,0.35) -- (axis cs:0.195,0.55,0.35) -- (axis cs:0.19,0.5449999999999999,0.35) -- (axis cs:0.185,0.54,0.35) -- (axis cs:0.185,0.53,0.35) -- (axis cs:0.185,0.52,0.35) -- (axis cs:0.18,0.515,0.35) -- (axis cs:0.175,0.51,0.35) -- (axis cs:0.175,0.5,0.35) -- (axis cs:0.175,0.49,0.35) -- (axis cs:0.175,0.48,0.35) -- (axis cs:0.17,0.475,0.35) -- (axis cs:0.165,0.47,0.35) -- (axis cs:0.165,0.45999999999999996,0.35) -- (axis cs:0.165,0.44999999999999996,0.35) -- (axis cs:0.165,0.43999999999999995,0.35) -- (axis cs:0.165,0.43000000000000005,0.35) -- (axis cs:0.16,0.42500000000000004,0.35) -- (axis cs:0.155,0.42000000000000004,0.35) -- (axis cs:0.155,0.41000000000000003,0.35) -- (axis cs:0.155,0.4,0.35) -- (axis cs:0.155,0.39,0.35) -- (axis cs:0.155,0.38,0.35) -- (axis cs:0.155,0.37,0.35) -- (axis cs:0.155,0.36,0.35) -- (axis cs:0.15,0.355,0.35) -- (axis cs:0.145,0.35,0.35) -- (axis cs:0.15,0.345,0.35) -- (axis cs:0.16,0.345,0.35) -- (axis cs:0.17,0.345,0.35) -- (axis cs:0.18,0.345,0.35) -- (axis cs:0.19,0.345,0.35) -- (axis cs:0.2,0.345,0.35) -- (axis cs:0.21,0.345,0.35) -- (axis cs:0.22,0.345,0.35) -- (axis cs:0.23,0.345,0.35) -- (axis cs:0.24,0.345,0.35) -- (axis cs:0.25,0.345,0.35) -- (axis cs:0.26,0.345,0.35) -- (axis cs:0.27,0.345,0.35) -- (axis cs:0.28,0.345,0.35) -- (axis cs:0.29,0.345,0.35) -- (axis cs:0.3,0.345,0.35) -- (axis cs:0.31,0.345,0.35) -- (axis cs:0.32,0.345,0.35) -- (axis cs:0.33,0.345,0.35) -- (axis cs:0.34,0.345,0.35) -- (axis cs:0.35,0.345,0.35) -- (axis cs:0.36,0.345,0.35) -- (axis cs:0.37,0.345,0.35) -- (axis cs:0.38,0.345,0.35) -- (axis cs:0.39,0.345,0.35) -- (axis cs:0.4,0.345,0.35) -- (axis cs:0.41,0.345,0.35) -- (axis cs:0.415,0.35,0.35) -- (axis cs:0.415,0.36,0.35) -- (axis cs:0.415,0.37,0.35) -- (axis cs:0.415,0.38,0.35) -- (axis cs:0.415,0.39,0.35) -- (axis cs:0.415,0.4,0.35) -- (axis cs:0.415,0.41000000000000003,0.35) -- (axis cs:0.415,0.42000000000000004,0.35) -- (axis cs:0.42,0.42500000000000004,0.35) -- (axis cs:0.43,0.42500000000000004,0.35) -- (axis cs:0.435,0.43000000000000005,0.35) -- (axis cs:0.44,0.43500000000000005,0.35) -- (axis cs:0.45,0.43500000000000005,0.35) -- (axis cs:0.46,0.43500000000000005,0.35) -- (axis cs:0.465,0.43999999999999995,0.35) -- (axis cs:0.47,0.44499999999999995,0.35) -- (axis cs:0.475,0.44999999999999996,0.35) -- (axis cs:0.48,0.45499999999999996,0.35) -- (axis cs:0.49,0.45499999999999996,0.35) -- (axis cs:0.495,0.45999999999999996,0.35) -- (axis cs:0.5,0.46499999999999997,0.35) -- (axis cs:0.505,0.47,0.35) -- (axis cs:0.51,0.475,0.35) -- (axis cs:0.515,0.48,0.35) -- (axis cs:0.515,0.49,0.35) -- (axis cs:0.52,0.495,0.35) -- (axis cs:0.525,0.5,0.35) -- (axis cs:0.53,0.505,0.35) -- (axis cs:0.535,0.51,0.35) -- (axis cs:0.535,0.52,0.35) -- (axis cs:0.535,0.53,0.35) -- (axis cs:0.535,0.54,0.35) -- (axis cs:0.535,0.55,0.35) -- (axis cs:0.53,0.5549999999999999,0.35) -- (axis cs:0.525,0.56,0.35) -- (axis cs:0.52,0.565,0.35) -- (axis cs:0.51,0.565,0.35) -- (axis cs:0.5,0.565,0.35) -- (axis cs:0.495,0.5700000000000001,0.35) -- (axis cs:0.49,0.575,0.35) -- (axis cs:0.48,0.575,0.35) -- (axis cs:0.47,0.575,0.35) -- (axis cs:0.46,0.575,0.35) -- (axis cs:0.455,0.5800000000000001,0.35) -- (axis cs:0.455,0.5900000000000001,0.35) -- (axis cs:0.46,0.595,0.35) -- (axis cs:0.465,0.6,0.35) -- (axis cs:0.47,0.605,0.35) -- (axis cs:0.475,0.61,0.35) -- (axis cs:0.48,0.615,0.35) -- (axis cs:0.485,0.62,0.35) -- (axis cs:0.49,0.625,0.35) -- (axis cs:0.495,0.63,0.35) -- (axis cs:0.5,0.635,0.35) -- (axis cs:0.51,0.635,0.35) -- (axis cs:0.515,0.64,0.35) -- (axis cs:0.52,0.645,0.35) -- (axis cs:0.53,0.645,0.35) -- (axis cs:0.54,0.645,0.35) -- (axis cs:0.545,0.65,0.35) -- (axis cs:0.55,0.655,0.35) -- (axis cs:0.56,0.655,0.35) -- (axis cs:0.57,0.655,0.35) -- (axis cs:0.58,0.655,0.35) -- (axis cs:0.59,0.655,0.35) -- (axis cs:0.595,0.65,0.35) -- (axis cs:0.6,0.645,0.35) -- (axis cs:0.61,0.645,0.35) -- (axis cs:0.62,0.645,0.35) -- (axis cs:0.63,0.645,0.35) -- (axis cs:0.635,0.64,0.35) -- (axis cs:0.64,0.635,0.35) -- (axis cs:0.65,0.635,0.35) -- (axis cs:0.655,0.63,0.35) -- (axis cs:0.66,0.625,0.35) -- (axis cs:0.67,0.625,0.35) -- (axis cs:0.675,0.62,0.35) -- (axis cs:0.68,0.615,0.35) -- (axis cs:0.685,0.61,0.35) -- (axis cs:0.69,0.605,0.35) -- (axis cs:0.695,0.6,0.35) -- (axis cs:0.7,0.595,0.35) -- (axis cs:0.705,0.5900000000000001,0.35) -- (axis cs:0.71,0.585,0.35) -- (axis cs:0.715,0.5800000000000001,0.35) -- (axis cs:0.72,0.575,0.35) -- (axis cs:0.725,0.5700000000000001,0.35) -- (axis cs:0.73,0.565,0.35) -- (axis cs:0.735,0.56,0.35) -- (axis cs:0.74,0.5549999999999999,0.35) -- (axis cs:0.745,0.55,0.35) -- (axis cs:0.745,0.54,0.35) -- (axis cs:0.75,0.5349999999999999,0.35) -- (axis cs:0.755,0.53,0.35) -- (axis cs:0.755,0.52,0.35) -- (axis cs:0.76,0.515,0.35) -- (axis cs:0.765,0.51,0.35) -- (axis cs:0.765,0.5,0.35) -- (axis cs:0.77,0.495,0.35) -- (axis cs:0.775,0.49,0.35) -- (axis cs:0.775,0.48,0.35) -- (axis cs:0.78,0.475,0.35) -- (axis cs:0.785,0.47,0.35) -- (axis cs:0.785,0.45999999999999996,0.35) -- (axis cs:0.785,0.44999999999999996,0.35) -- (axis cs:0.79,0.44499999999999995,0.35) -- (axis cs:0.795,0.43999999999999995,0.35) -- (axis cs:0.795,0.43000000000000005,0.35) -- (axis cs:0.795,0.42000000000000004,0.35) -- (axis cs:0.8,0.41500000000000004,0.35) -- (axis cs:0.805,0.41000000000000003,0.35) -- (axis cs:0.805,0.4,0.35) -- (axis cs:0.805,0.39,0.35) -- (axis cs:0.805,0.38,0.35) -- (axis cs:0.81,0.375,0.35) -- (axis cs:0.815,0.37,0.35) -- (axis cs:0.815,0.36,0.35) -- (axis cs:0.815,0.35,0.35) -- (axis cs:0.82,0.345,0.35) -- (axis cs:0.83,0.345,0.35) -- (axis cs:0.84,0.345,0.35) -- (axis cs:0.85,0.345,0.35) -- (axis cs:0.86,0.345,0.35) -- (axis cs:0.87,0.345,0.35) -- (axis cs:0.88,0.345,0.35) -- (axis cs:0.89,0.345,0.35) -- (axis cs:0.9,0.345,0.35) -- (axis cs:0.91,0.345,0.35) -- (axis cs:0.92,0.345,0.35) -- (axis cs:0.93,0.345,0.35) -- (axis cs:0.94,0.345,0.35) -- (axis cs:0.95,0.345,0.35) -- (axis cs:0.96,0.345,0.35) -- (axis cs:0.97,0.345,0.35) -- (axis cs:0.98,0.345,0.35) -- (axis cs:0.99,0.345,0.35) -- (axis cs:1.0,0.345,0.35) -- (axis cs:1.0,0.0,0.35) -- (axis cs:0.0,0.0,0.35) -- cycle
;

      \draw[ultra thick,black] (axis cs:0.0,1.0,0.35) -- (axis cs:1.0,1.0,0.35) -- (axis cs:1.0,0.0,0.35);
   \end{scope}
   
   % top of RxO (back, positive y half, in green box)
   \addplot3[
           surf,
           opacity = 1.0,
           samples=9,
           samples y=7,
           domain=0.8:1.0,y domain=0:1,
           z buffer=sort]
       ({x},
        {(0.5+0.15*(y))},
        {((8*x^3 - 11*x^2 + 4*x)*(1-(0.5+0.15*(y))^2) + (x)*((0.5+0.15*(y))^2)) + 0.05*(max(0.00001,1-(2*y)^2-2*(3*(x-0.5))^2))^0.5});
   
   % part of pbottom manifold IN FRONT OF RxO, with boundary
   \begin{scope}
      \clip (axis cs:0.8,1.0,0.35) -- (axis cs:1.0,1.0,0.35) -- (axis cs:1.0,0.5,0.35) -- (axis cs:0.8,0.5,0.35) -- cycle;
      \fill[red!50,opacity=0.5,even odd rule]
(axis cs:0.0,0.655,0.35) -- (axis cs:0.01,0.655,0.35) -- (axis cs:0.02,0.655,0.35) -- (axis cs:0.03,0.655,0.35) -- (axis cs:0.04,0.655,0.35) -- (axis cs:0.05,0.655,0.35) -- (axis cs:0.06,0.655,0.35) -- (axis cs:0.07,0.655,0.35) -- (axis cs:0.08,0.655,0.35) -- (axis cs:0.09,0.655,0.35) -- (axis cs:0.1,0.655,0.35) -- (axis cs:0.11,0.655,0.35) -- (axis cs:0.12,0.655,0.35) -- (axis cs:0.13,0.655,0.35) -- (axis cs:0.14,0.655,0.35) -- (axis cs:0.15,0.655,0.35) -- (axis cs:0.16,0.655,0.35) -- (axis cs:0.17,0.655,0.35) -- (axis cs:0.18,0.655,0.35) -- (axis cs:0.19,0.655,0.35) -- (axis cs:0.2,0.655,0.35) -- (axis cs:0.21,0.655,0.35) -- (axis cs:0.22,0.655,0.35) -- (axis cs:0.23,0.655,0.35) -- (axis cs:0.235,0.65,0.35) -- (axis cs:0.23,0.645,0.35) -- (axis cs:0.225,0.64,0.35) -- (axis cs:0.225,0.63,0.35) -- (axis cs:0.22,0.625,0.35) -- (axis cs:0.215,0.62,0.35) -- (axis cs:0.215,0.61,0.35) -- (axis cs:0.21,0.605,0.35) -- (axis cs:0.205,0.6,0.35) -- (axis cs:0.205,0.5900000000000001,0.35) -- (axis cs:0.205,0.5800000000000001,0.35) -- (axis cs:0.2,0.575,0.35) -- (axis cs:0.195,0.5700000000000001,0.35) -- (axis cs:0.195,0.56,0.35) -- (axis cs:0.195,0.55,0.35) -- (axis cs:0.19,0.5449999999999999,0.35) -- (axis cs:0.185,0.54,0.35) -- (axis cs:0.185,0.53,0.35) -- (axis cs:0.185,0.52,0.35) -- (axis cs:0.18,0.515,0.35) -- (axis cs:0.175,0.51,0.35) -- (axis cs:0.175,0.5,0.35) -- (axis cs:0.175,0.49,0.35) -- (axis cs:0.175,0.48,0.35) -- (axis cs:0.17,0.475,0.35) -- (axis cs:0.165,0.47,0.35) -- (axis cs:0.165,0.45999999999999996,0.35) -- (axis cs:0.165,0.44999999999999996,0.35) -- (axis cs:0.165,0.43999999999999995,0.35) -- (axis cs:0.165,0.43000000000000005,0.35) -- (axis cs:0.16,0.42500000000000004,0.35) -- (axis cs:0.155,0.42000000000000004,0.35) -- (axis cs:0.155,0.41000000000000003,0.35) -- (axis cs:0.155,0.4,0.35) -- (axis cs:0.155,0.39,0.35) -- (axis cs:0.155,0.38,0.35) -- (axis cs:0.155,0.37,0.35) -- (axis cs:0.155,0.36,0.35) -- (axis cs:0.15,0.355,0.35) -- (axis cs:0.145,0.35,0.35) -- (axis cs:0.15,0.345,0.35) -- (axis cs:0.16,0.345,0.35) -- (axis cs:0.17,0.345,0.35) -- (axis cs:0.18,0.345,0.35) -- (axis cs:0.19,0.345,0.35) -- (axis cs:0.2,0.345,0.35) -- (axis cs:0.21,0.345,0.35) -- (axis cs:0.22,0.345,0.35) -- (axis cs:0.23,0.345,0.35) -- (axis cs:0.24,0.345,0.35) -- (axis cs:0.25,0.345,0.35) -- (axis cs:0.26,0.345,0.35) -- (axis cs:0.27,0.345,0.35) -- (axis cs:0.28,0.345,0.35) -- (axis cs:0.29,0.345,0.35) -- (axis cs:0.3,0.345,0.35) -- (axis cs:0.31,0.345,0.35) -- (axis cs:0.32,0.345,0.35) -- (axis cs:0.33,0.345,0.35) -- (axis cs:0.34,0.345,0.35) -- (axis cs:0.35,0.345,0.35) -- (axis cs:0.36,0.345,0.35) -- (axis cs:0.37,0.345,0.35) -- (axis cs:0.38,0.345,0.35) -- (axis cs:0.39,0.345,0.35) -- (axis cs:0.4,0.345,0.35) -- (axis cs:0.41,0.345,0.35) -- (axis cs:0.415,0.35,0.35) -- (axis cs:0.415,0.36,0.35) -- (axis cs:0.415,0.37,0.35) -- (axis cs:0.415,0.38,0.35) -- (axis cs:0.415,0.39,0.35) -- (axis cs:0.415,0.4,0.35) -- (axis cs:0.415,0.41000000000000003,0.35) -- (axis cs:0.415,0.42000000000000004,0.35) -- (axis cs:0.42,0.42500000000000004,0.35) -- (axis cs:0.43,0.42500000000000004,0.35) -- (axis cs:0.435,0.43000000000000005,0.35) -- (axis cs:0.44,0.43500000000000005,0.35) -- (axis cs:0.45,0.43500000000000005,0.35) -- (axis cs:0.46,0.43500000000000005,0.35) -- (axis cs:0.465,0.43999999999999995,0.35) -- (axis cs:0.47,0.44499999999999995,0.35) -- (axis cs:0.475,0.44999999999999996,0.35) -- (axis cs:0.48,0.45499999999999996,0.35) -- (axis cs:0.49,0.45499999999999996,0.35) -- (axis cs:0.495,0.45999999999999996,0.35) -- (axis cs:0.5,0.46499999999999997,0.35) -- (axis cs:0.505,0.47,0.35) -- (axis cs:0.51,0.475,0.35) -- (axis cs:0.515,0.48,0.35) -- (axis cs:0.515,0.49,0.35) -- (axis cs:0.52,0.495,0.35) -- (axis cs:0.525,0.5,0.35) -- (axis cs:0.53,0.505,0.35) -- (axis cs:0.535,0.51,0.35) -- (axis cs:0.535,0.52,0.35) -- (axis cs:0.535,0.53,0.35) -- (axis cs:0.535,0.54,0.35) -- (axis cs:0.535,0.55,0.35) -- (axis cs:0.53,0.5549999999999999,0.35) -- (axis cs:0.525,0.56,0.35) -- (axis cs:0.52,0.565,0.35) -- (axis cs:0.51,0.565,0.35) -- (axis cs:0.5,0.565,0.35) -- (axis cs:0.495,0.5700000000000001,0.35) -- (axis cs:0.49,0.575,0.35) -- (axis cs:0.48,0.575,0.35) -- (axis cs:0.47,0.575,0.35) -- (axis cs:0.46,0.575,0.35) -- (axis cs:0.455,0.5800000000000001,0.35) -- (axis cs:0.455,0.5900000000000001,0.35) -- (axis cs:0.46,0.595,0.35) -- (axis cs:0.465,0.6,0.35) -- (axis cs:0.47,0.605,0.35) -- (axis cs:0.475,0.61,0.35) -- (axis cs:0.48,0.615,0.35) -- (axis cs:0.485,0.62,0.35) -- (axis cs:0.49,0.625,0.35) -- (axis cs:0.495,0.63,0.35) -- (axis cs:0.5,0.635,0.35) -- (axis cs:0.51,0.635,0.35) -- (axis cs:0.515,0.64,0.35) -- (axis cs:0.52,0.645,0.35) -- (axis cs:0.53,0.645,0.35) -- (axis cs:0.54,0.645,0.35) -- (axis cs:0.545,0.65,0.35) -- (axis cs:0.55,0.655,0.35) -- (axis cs:0.56,0.655,0.35) -- (axis cs:0.57,0.655,0.35) -- (axis cs:0.58,0.655,0.35) -- (axis cs:0.59,0.655,0.35) -- (axis cs:0.595,0.65,0.35) -- (axis cs:0.6,0.645,0.35) -- (axis cs:0.61,0.645,0.35) -- (axis cs:0.62,0.645,0.35) -- (axis cs:0.63,0.645,0.35) -- (axis cs:0.635,0.64,0.35) -- (axis cs:0.64,0.635,0.35) -- (axis cs:0.65,0.635,0.35) -- (axis cs:0.655,0.63,0.35) -- (axis cs:0.66,0.625,0.35) -- (axis cs:0.67,0.625,0.35) -- (axis cs:0.675,0.62,0.35) -- (axis cs:0.68,0.615,0.35) -- (axis cs:0.685,0.61,0.35) -- (axis cs:0.69,0.605,0.35) -- (axis cs:0.695,0.6,0.35) -- (axis cs:0.7,0.595,0.35) -- (axis cs:0.705,0.5900000000000001,0.35) -- (axis cs:0.71,0.585,0.35) -- (axis cs:0.715,0.5800000000000001,0.35) -- (axis cs:0.72,0.575,0.35) -- (axis cs:0.725,0.5700000000000001,0.35) -- (axis cs:0.73,0.565,0.35) -- (axis cs:0.735,0.56,0.35) -- (axis cs:0.74,0.5549999999999999,0.35) -- (axis cs:0.745,0.55,0.35) -- (axis cs:0.745,0.54,0.35) -- (axis cs:0.75,0.5349999999999999,0.35) -- (axis cs:0.755,0.53,0.35) -- (axis cs:0.755,0.52,0.35) -- (axis cs:0.76,0.515,0.35) -- (axis cs:0.765,0.51,0.35) -- (axis cs:0.765,0.5,0.35) -- (axis cs:0.77,0.495,0.35) -- (axis cs:0.775,0.49,0.35) -- (axis cs:0.775,0.48,0.35) -- (axis cs:0.78,0.475,0.35) -- (axis cs:0.785,0.47,0.35) -- (axis cs:0.785,0.45999999999999996,0.35) -- (axis cs:0.785,0.44999999999999996,0.35) -- (axis cs:0.79,0.44499999999999995,0.35) -- (axis cs:0.795,0.43999999999999995,0.35) -- (axis cs:0.795,0.43000000000000005,0.35) -- (axis cs:0.795,0.42000000000000004,0.35) -- (axis cs:0.8,0.41500000000000004,0.35) -- (axis cs:0.805,0.41000000000000003,0.35) -- (axis cs:0.805,0.4,0.35) -- (axis cs:0.805,0.39,0.35) -- (axis cs:0.805,0.38,0.35) -- (axis cs:0.81,0.375,0.35) -- (axis cs:0.815,0.37,0.35) -- (axis cs:0.815,0.36,0.35) -- (axis cs:0.815,0.35,0.35) -- (axis cs:0.82,0.345,0.35) -- (axis cs:0.83,0.345,0.35) -- (axis cs:0.84,0.345,0.35) -- (axis cs:0.85,0.345,0.35) -- (axis cs:0.86,0.345,0.35) -- (axis cs:0.87,0.345,0.35) -- (axis cs:0.88,0.345,0.35) -- (axis cs:0.89,0.345,0.35) -- (axis cs:0.9,0.345,0.35) -- (axis cs:0.91,0.345,0.35) -- (axis cs:0.92,0.345,0.35) -- (axis cs:0.93,0.345,0.35) -- (axis cs:0.94,0.345,0.35) -- (axis cs:0.95,0.345,0.35) -- (axis cs:0.96,0.345,0.35) -- (axis cs:0.97,0.345,0.35) -- (axis cs:0.98,0.345,0.35) -- (axis cs:0.99,0.345,0.35) -- (axis cs:1.0,0.345,0.35) -- (axis cs:1.0,0.0,0.35) -- (axis cs:0.0,0.0,0.35) -- cycle
;

   \end{scope}
   
   
   %%% next, draw green box

   % RxRE side surface (domain is z)
   %\addplot3[
   %   surf,green!50!white,faceted color=green!30!black,
   %   opacity=0.5,
   %   samples=21,samples y=21,
   %   domain=0:1,y domain=0.5:1.0,
   %   z buffer=sort]
   %   ({0.8}, {y}, {x});
   \fill[green!50,opacity=0.5]
      (axis cs:0.8,1.0,1.0) -- (axis cs:0.8,0.5,1.0) -- (axis cs:0.8,0.5,0.0) -- (axis cs:0.8,1.0,0.0) -- cycle;
   % RxRE front surface (domain is z)
   %\addplot3[
   %   surf,green!50!white,faceted color=green!50!black,
   %   opacity=0.5,
   %   samples=21,samples y=9,
   %   domain=0:1,y domain=0.8:1.0,
   %   z buffer=sort]
   %   ({y}, {0.5}, {x});
   \fill[green!50,opacity=0.5]
      (axis cs:0.8,0.5,1.0) -- (axis cs:1.0,0.5,1.0) -- (axis cs:1.0,0.5,0.0) -- (axis cs:0.8,0.5,0.0) -- cycle;
   % RxRE top surface
   %\addplot3[
   %   surf,green!50!white,faceted color=green!50!black,
   %   opacity=0.5,
   %   samples=9,samples y=21,
   %   domain=0.8:1,y domain=0.5:1.0,
   %   z buffer=sort]
   %   ({x}, {y}, {1.0});
   \fill[green!50,opacity=0.5]
      (axis cs:0.8,1.0,1.0) -- (axis cs:1.0,1.0,1.0) -- (axis cs:1.0,0.5,1.0) -- (axis cs:0.8,0.5,1.0) -- cycle;
   % add lines (top)
   %\draw[green!50!black] ;
   % add lines (sides)
   \draw[green!50!black]
      (axis cs:0.8,1.0,1.0) -- (axis cs:1.0,1.0,1.0) -- (axis cs:1.0,0.5,1.0) -- (axis cs:0.8,0.5,1.0) -- cycle
      (axis cs:0.8,1.0,0.0) -- (axis cs:0.8,1.0,1.0)
      (axis cs:1.0,1.0,0.0) -- (axis cs:1.0,1.0,1.0)
      (axis cs:0.8,0.5,0.0) -- (axis cs:0.8,0.5,1.0)
      (axis cs:1.0,0.5,0.0) -- (axis cs:1.0,0.5,1.0);

   % fix up side/front surfaces with transparent blue
%   \fill[blue!50,opacity=0.5]
%      (axis cs:0.8,1,0) -- (axis cs:0.8,1,0.25) -- (axis cs:0.8,0.5,0.25) -- (axis cs:0.8,0.5,0) -- cycle;
%   \fill[blue!50,opacity=0.5]
%      (axis cs:0.8,0.5,0.07) -- (axis cs:0.8,0.5,0.25) -- (axis cs:1.0,0.5,0.25) -- (axis cs:1.0,0.5,0.07) -- cycle;

   %%% next, draw stuff OUTSIDE the green box
   
   % part of eo top surface BEHIND (or INSIDE OF) RxO
   \begin{scope}
      \clip (axis cs:0.0,1.0,0.25) -- (axis cs:0.8,1.0,0.25) -- (axis cs:0.8,0.5,0.25) -- (axis cs:1.0,0.5,0.25)
         -- (axis cs:1.0,0.0,0.25) -- (axis cs:0.0,0.0,0.25) -- cycle;
      \fill[blue!50,opacity=0.5,even odd rule]
(axis cs:0,0,0.25) -- (axis cs:1,0,0.25) -- (axis cs:1,1,0.25) -- (axis cs:0,1,0.25) -- cycle
(axis cs:0.0,0.655,0.25) -- (axis cs:0.01,0.655,0.25) -- (axis cs:0.02,0.655,0.25) -- (axis cs:0.03,0.655,0.25) -- (axis cs:0.04,0.655,0.25) -- (axis cs:0.05,0.655,0.25) -- (axis cs:0.06,0.655,0.25) -- (axis cs:0.07,0.655,0.25) -- (axis cs:0.08,0.655,0.25) -- (axis cs:0.09,0.655,0.25) -- (axis cs:0.1,0.655,0.25) -- (axis cs:0.11,0.655,0.25) -- (axis cs:0.12,0.655,0.25) -- (axis cs:0.125,0.65,0.25) -- (axis cs:0.125,0.64,0.25) -- (axis cs:0.125,0.63,0.25) -- (axis cs:0.12,0.625,0.25) -- (axis cs:0.115,0.62,0.25) -- (axis cs:0.115,0.61,0.25) -- (axis cs:0.115,0.6,0.25) -- (axis cs:0.115,0.5900000000000001,0.25) -- (axis cs:0.115,0.5800000000000001,0.25) -- (axis cs:0.11,0.575,0.25) -- (axis cs:0.105,0.5700000000000001,0.25) -- (axis cs:0.105,0.56,0.25) -- (axis cs:0.105,0.55,0.25) -- (axis cs:0.105,0.54,0.25) -- (axis cs:0.105,0.53,0.25) -- (axis cs:0.105,0.52,0.25) -- (axis cs:0.105,0.51,0.25) -- (axis cs:0.105,0.5,0.25) -- (axis cs:0.1,0.495,0.25) -- (axis cs:0.095,0.49,0.25) -- (axis cs:0.095,0.48,0.25) -- (axis cs:0.095,0.47,0.25) -- (axis cs:0.095,0.45999999999999996,0.25) -- (axis cs:0.095,0.44999999999999996,0.25) -- (axis cs:0.095,0.43999999999999995,0.25) -- (axis cs:0.095,0.43000000000000005,0.25) -- (axis cs:0.095,0.42000000000000004,0.25) -- (axis cs:0.095,0.41000000000000003,0.25) -- (axis cs:0.095,0.4,0.25) -- (axis cs:0.095,0.39,0.25) -- (axis cs:0.095,0.38,0.25) -- (axis cs:0.09,0.375,0.25) -- (axis cs:0.085,0.37,0.25) -- (axis cs:0.085,0.36,0.25) -- (axis cs:0.085,0.35,0.25) -- (axis cs:0.09,0.345,0.25) -- (axis cs:0.1,0.345,0.25) -- (axis cs:0.11,0.345,0.25) -- (axis cs:0.12,0.345,0.25) -- (axis cs:0.13,0.345,0.25) -- (axis cs:0.14,0.345,0.25) -- (axis cs:0.15,0.345,0.25) -- (axis cs:0.16,0.345,0.25) -- (axis cs:0.17,0.345,0.25) -- (axis cs:0.18,0.345,0.25) -- (axis cs:0.19,0.345,0.25) -- (axis cs:0.2,0.345,0.25) -- (axis cs:0.21,0.345,0.25) -- (axis cs:0.22,0.345,0.25) -- (axis cs:0.23,0.345,0.25) -- (axis cs:0.24,0.345,0.25) -- (axis cs:0.25,0.345,0.25) -- (axis cs:0.26,0.345,0.25) -- (axis cs:0.27,0.345,0.25) -- (axis cs:0.28,0.345,0.25) -- (axis cs:0.29,0.345,0.25) -- (axis cs:0.3,0.345,0.25) -- (axis cs:0.31,0.345,0.25) -- (axis cs:0.32,0.345,0.25) -- (axis cs:0.33,0.345,0.25) -- (axis cs:0.34,0.345,0.25) -- (axis cs:0.35,0.345,0.25) -- (axis cs:0.36,0.345,0.25) -- (axis cs:0.37,0.345,0.25) -- (axis cs:0.38,0.345,0.25) -- (axis cs:0.39,0.345,0.25) -- (axis cs:0.4,0.345,0.25) -- (axis cs:0.41,0.345,0.25) -- (axis cs:0.42,0.345,0.25) -- (axis cs:0.43,0.345,0.25) -- (axis cs:0.44,0.345,0.25) -- (axis cs:0.45,0.345,0.25) -- (axis cs:0.46,0.345,0.25) -- (axis cs:0.47,0.345,0.25) -- (axis cs:0.48,0.345,0.25) -- (axis cs:0.49,0.345,0.25) -- (axis cs:0.5,0.345,0.25) -- (axis cs:0.51,0.345,0.25) -- (axis cs:0.52,0.345,0.25) -- (axis cs:0.53,0.345,0.25) -- (axis cs:0.54,0.345,0.25) -- (axis cs:0.545,0.35,0.25) -- (axis cs:0.545,0.36,0.25) -- (axis cs:0.55,0.365,0.25) -- (axis cs:0.555,0.37,0.25) -- (axis cs:0.555,0.38,0.25) -- (axis cs:0.56,0.385,0.25) -- (axis cs:0.565,0.39,0.25) -- (axis cs:0.565,0.4,0.25) -- (axis cs:0.57,0.405,0.25) -- (axis cs:0.575,0.41000000000000003,0.25) -- (axis cs:0.58,0.41500000000000004,0.25) -- (axis cs:0.585,0.42000000000000004,0.25) -- (axis cs:0.59,0.42500000000000004,0.25) -- (axis cs:0.595,0.43000000000000005,0.25) -- (axis cs:0.6,0.43500000000000005,0.25) -- (axis cs:0.61,0.43500000000000005,0.25) -- (axis cs:0.62,0.43500000000000005,0.25) -- (axis cs:0.63,0.43500000000000005,0.25) -- (axis cs:0.64,0.43500000000000005,0.25) -- (axis cs:0.645,0.43999999999999995,0.25) -- (axis cs:0.65,0.44499999999999995,0.25) -- (axis cs:0.66,0.44499999999999995,0.25) -- (axis cs:0.67,0.44499999999999995,0.25) -- (axis cs:0.675,0.43999999999999995,0.25) -- (axis cs:0.68,0.43500000000000005,0.25) -- (axis cs:0.685,0.43000000000000005,0.25) -- (axis cs:0.69,0.42500000000000004,0.25) -- (axis cs:0.695,0.42000000000000004,0.25) -- (axis cs:0.7,0.41500000000000004,0.25) -- (axis cs:0.705,0.41000000000000003,0.25) -- (axis cs:0.71,0.405,0.25) -- (axis cs:0.715,0.4,0.25) -- (axis cs:0.72,0.395,0.25) -- (axis cs:0.725,0.39,0.25) -- (axis cs:0.725,0.38,0.25) -- (axis cs:0.73,0.375,0.25) -- (axis cs:0.735,0.37,0.25) -- (axis cs:0.735,0.36,0.25) -- (axis cs:0.74,0.355,0.25) -- (axis cs:0.745,0.35,0.25) -- (axis cs:0.75,0.345,0.25) -- (axis cs:0.76,0.345,0.25) -- (axis cs:0.77,0.345,0.25) -- (axis cs:0.78,0.345,0.25) -- (axis cs:0.79,0.345,0.25) -- (axis cs:0.8,0.345,0.25) -- (axis cs:0.81,0.345,0.25) -- (axis cs:0.82,0.345,0.25) -- (axis cs:0.83,0.345,0.25) -- (axis cs:0.84,0.345,0.25) -- (axis cs:0.85,0.345,0.25) -- (axis cs:0.86,0.345,0.25) -- (axis cs:0.87,0.345,0.25) -- (axis cs:0.88,0.345,0.25) -- (axis cs:0.89,0.345,0.25) -- (axis cs:0.9,0.345,0.25) -- (axis cs:0.91,0.345,0.25) -- (axis cs:0.92,0.345,0.25) -- (axis cs:0.93,0.345,0.25) -- (axis cs:0.94,0.345,0.25) -- (axis cs:0.95,0.345,0.25) -- (axis cs:0.96,0.345,0.25) -- (axis cs:0.97,0.345,0.25) -- (axis cs:0.98,0.345,0.25) -- (axis cs:0.99,0.345,0.25) -- (axis cs:1.0,0.345,0.25) -- (axis cs:1.0,0.0,0.25) -- (axis cs:0.0,0.0,0.25) -- cycle
;

   \end{scope}

   % part of pbottom manifold BEHIND (or INSIDE OF) RxO
   \begin{scope}
      \clip (axis cs:0.0,1.0,0.35) -- (axis cs:0.8,1.0,0.35) -- (axis cs:0.8,0.5,0.35) -- (axis cs:1.0,0.5,0.35)
         -- (axis cs:1.0,0.0,0.35) -- (axis cs:0.0,0.0,0.35) -- cycle;
      \fill[red!50,opacity=0.5,even odd rule]
(axis cs:0,0,0.35) -- (axis cs:1,0,0.35) -- (axis cs:1,1,0.35) -- (axis cs:0,1,0.35) -- cycle
(axis cs:0.0,0.655,0.35) -- (axis cs:0.01,0.655,0.35) -- (axis cs:0.02,0.655,0.35) -- (axis cs:0.03,0.655,0.35) -- (axis cs:0.04,0.655,0.35) -- (axis cs:0.05,0.655,0.35) -- (axis cs:0.06,0.655,0.35) -- (axis cs:0.07,0.655,0.35) -- (axis cs:0.08,0.655,0.35) -- (axis cs:0.09,0.655,0.35) -- (axis cs:0.1,0.655,0.35) -- (axis cs:0.11,0.655,0.35) -- (axis cs:0.12,0.655,0.35) -- (axis cs:0.13,0.655,0.35) -- (axis cs:0.14,0.655,0.35) -- (axis cs:0.15,0.655,0.35) -- (axis cs:0.16,0.655,0.35) -- (axis cs:0.17,0.655,0.35) -- (axis cs:0.18,0.655,0.35) -- (axis cs:0.19,0.655,0.35) -- (axis cs:0.2,0.655,0.35) -- (axis cs:0.21,0.655,0.35) -- (axis cs:0.22,0.655,0.35) -- (axis cs:0.23,0.655,0.35) -- (axis cs:0.235,0.65,0.35) -- (axis cs:0.23,0.645,0.35) -- (axis cs:0.225,0.64,0.35) -- (axis cs:0.225,0.63,0.35) -- (axis cs:0.22,0.625,0.35) -- (axis cs:0.215,0.62,0.35) -- (axis cs:0.215,0.61,0.35) -- (axis cs:0.21,0.605,0.35) -- (axis cs:0.205,0.6,0.35) -- (axis cs:0.205,0.5900000000000001,0.35) -- (axis cs:0.205,0.5800000000000001,0.35) -- (axis cs:0.2,0.575,0.35) -- (axis cs:0.195,0.5700000000000001,0.35) -- (axis cs:0.195,0.56,0.35) -- (axis cs:0.195,0.55,0.35) -- (axis cs:0.19,0.5449999999999999,0.35) -- (axis cs:0.185,0.54,0.35) -- (axis cs:0.185,0.53,0.35) -- (axis cs:0.185,0.52,0.35) -- (axis cs:0.18,0.515,0.35) -- (axis cs:0.175,0.51,0.35) -- (axis cs:0.175,0.5,0.35) -- (axis cs:0.175,0.49,0.35) -- (axis cs:0.175,0.48,0.35) -- (axis cs:0.17,0.475,0.35) -- (axis cs:0.165,0.47,0.35) -- (axis cs:0.165,0.45999999999999996,0.35) -- (axis cs:0.165,0.44999999999999996,0.35) -- (axis cs:0.165,0.43999999999999995,0.35) -- (axis cs:0.165,0.43000000000000005,0.35) -- (axis cs:0.16,0.42500000000000004,0.35) -- (axis cs:0.155,0.42000000000000004,0.35) -- (axis cs:0.155,0.41000000000000003,0.35) -- (axis cs:0.155,0.4,0.35) -- (axis cs:0.155,0.39,0.35) -- (axis cs:0.155,0.38,0.35) -- (axis cs:0.155,0.37,0.35) -- (axis cs:0.155,0.36,0.35) -- (axis cs:0.15,0.355,0.35) -- (axis cs:0.145,0.35,0.35) -- (axis cs:0.15,0.345,0.35) -- (axis cs:0.16,0.345,0.35) -- (axis cs:0.17,0.345,0.35) -- (axis cs:0.18,0.345,0.35) -- (axis cs:0.19,0.345,0.35) -- (axis cs:0.2,0.345,0.35) -- (axis cs:0.21,0.345,0.35) -- (axis cs:0.22,0.345,0.35) -- (axis cs:0.23,0.345,0.35) -- (axis cs:0.24,0.345,0.35) -- (axis cs:0.25,0.345,0.35) -- (axis cs:0.26,0.345,0.35) -- (axis cs:0.27,0.345,0.35) -- (axis cs:0.28,0.345,0.35) -- (axis cs:0.29,0.345,0.35) -- (axis cs:0.3,0.345,0.35) -- (axis cs:0.31,0.345,0.35) -- (axis cs:0.32,0.345,0.35) -- (axis cs:0.33,0.345,0.35) -- (axis cs:0.34,0.345,0.35) -- (axis cs:0.35,0.345,0.35) -- (axis cs:0.36,0.345,0.35) -- (axis cs:0.37,0.345,0.35) -- (axis cs:0.38,0.345,0.35) -- (axis cs:0.39,0.345,0.35) -- (axis cs:0.4,0.345,0.35) -- (axis cs:0.41,0.345,0.35) -- (axis cs:0.415,0.35,0.35) -- (axis cs:0.415,0.36,0.35) -- (axis cs:0.415,0.37,0.35) -- (axis cs:0.415,0.38,0.35) -- (axis cs:0.415,0.39,0.35) -- (axis cs:0.415,0.4,0.35) -- (axis cs:0.415,0.41000000000000003,0.35) -- (axis cs:0.415,0.42000000000000004,0.35) -- (axis cs:0.42,0.42500000000000004,0.35) -- (axis cs:0.43,0.42500000000000004,0.35) -- (axis cs:0.435,0.43000000000000005,0.35) -- (axis cs:0.44,0.43500000000000005,0.35) -- (axis cs:0.45,0.43500000000000005,0.35) -- (axis cs:0.46,0.43500000000000005,0.35) -- (axis cs:0.465,0.43999999999999995,0.35) -- (axis cs:0.47,0.44499999999999995,0.35) -- (axis cs:0.475,0.44999999999999996,0.35) -- (axis cs:0.48,0.45499999999999996,0.35) -- (axis cs:0.49,0.45499999999999996,0.35) -- (axis cs:0.495,0.45999999999999996,0.35) -- (axis cs:0.5,0.46499999999999997,0.35) -- (axis cs:0.505,0.47,0.35) -- (axis cs:0.51,0.475,0.35) -- (axis cs:0.515,0.48,0.35) -- (axis cs:0.515,0.49,0.35) -- (axis cs:0.52,0.495,0.35) -- (axis cs:0.525,0.5,0.35) -- (axis cs:0.53,0.505,0.35) -- (axis cs:0.535,0.51,0.35) -- (axis cs:0.535,0.52,0.35) -- (axis cs:0.535,0.53,0.35) -- (axis cs:0.535,0.54,0.35) -- (axis cs:0.535,0.55,0.35) -- (axis cs:0.53,0.5549999999999999,0.35) -- (axis cs:0.525,0.56,0.35) -- (axis cs:0.52,0.565,0.35) -- (axis cs:0.51,0.565,0.35) -- (axis cs:0.5,0.565,0.35) -- (axis cs:0.495,0.5700000000000001,0.35) -- (axis cs:0.49,0.575,0.35) -- (axis cs:0.48,0.575,0.35) -- (axis cs:0.47,0.575,0.35) -- (axis cs:0.46,0.575,0.35) -- (axis cs:0.455,0.5800000000000001,0.35) -- (axis cs:0.455,0.5900000000000001,0.35) -- (axis cs:0.46,0.595,0.35) -- (axis cs:0.465,0.6,0.35) -- (axis cs:0.47,0.605,0.35) -- (axis cs:0.475,0.61,0.35) -- (axis cs:0.48,0.615,0.35) -- (axis cs:0.485,0.62,0.35) -- (axis cs:0.49,0.625,0.35) -- (axis cs:0.495,0.63,0.35) -- (axis cs:0.5,0.635,0.35) -- (axis cs:0.51,0.635,0.35) -- (axis cs:0.515,0.64,0.35) -- (axis cs:0.52,0.645,0.35) -- (axis cs:0.53,0.645,0.35) -- (axis cs:0.54,0.645,0.35) -- (axis cs:0.545,0.65,0.35) -- (axis cs:0.55,0.655,0.35) -- (axis cs:0.56,0.655,0.35) -- (axis cs:0.57,0.655,0.35) -- (axis cs:0.58,0.655,0.35) -- (axis cs:0.59,0.655,0.35) -- (axis cs:0.595,0.65,0.35) -- (axis cs:0.6,0.645,0.35) -- (axis cs:0.61,0.645,0.35) -- (axis cs:0.62,0.645,0.35) -- (axis cs:0.63,0.645,0.35) -- (axis cs:0.635,0.64,0.35) -- (axis cs:0.64,0.635,0.35) -- (axis cs:0.65,0.635,0.35) -- (axis cs:0.655,0.63,0.35) -- (axis cs:0.66,0.625,0.35) -- (axis cs:0.67,0.625,0.35) -- (axis cs:0.675,0.62,0.35) -- (axis cs:0.68,0.615,0.35) -- (axis cs:0.685,0.61,0.35) -- (axis cs:0.69,0.605,0.35) -- (axis cs:0.695,0.6,0.35) -- (axis cs:0.7,0.595,0.35) -- (axis cs:0.705,0.5900000000000001,0.35) -- (axis cs:0.71,0.585,0.35) -- (axis cs:0.715,0.5800000000000001,0.35) -- (axis cs:0.72,0.575,0.35) -- (axis cs:0.725,0.5700000000000001,0.35) -- (axis cs:0.73,0.565,0.35) -- (axis cs:0.735,0.56,0.35) -- (axis cs:0.74,0.5549999999999999,0.35) -- (axis cs:0.745,0.55,0.35) -- (axis cs:0.745,0.54,0.35) -- (axis cs:0.75,0.5349999999999999,0.35) -- (axis cs:0.755,0.53,0.35) -- (axis cs:0.755,0.52,0.35) -- (axis cs:0.76,0.515,0.35) -- (axis cs:0.765,0.51,0.35) -- (axis cs:0.765,0.5,0.35) -- (axis cs:0.77,0.495,0.35) -- (axis cs:0.775,0.49,0.35) -- (axis cs:0.775,0.48,0.35) -- (axis cs:0.78,0.475,0.35) -- (axis cs:0.785,0.47,0.35) -- (axis cs:0.785,0.45999999999999996,0.35) -- (axis cs:0.785,0.44999999999999996,0.35) -- (axis cs:0.79,0.44499999999999995,0.35) -- (axis cs:0.795,0.43999999999999995,0.35) -- (axis cs:0.795,0.43000000000000005,0.35) -- (axis cs:0.795,0.42000000000000004,0.35) -- (axis cs:0.8,0.41500000000000004,0.35) -- (axis cs:0.805,0.41000000000000003,0.35) -- (axis cs:0.805,0.4,0.35) -- (axis cs:0.805,0.39,0.35) -- (axis cs:0.805,0.38,0.35) -- (axis cs:0.81,0.375,0.35) -- (axis cs:0.815,0.37,0.35) -- (axis cs:0.815,0.36,0.35) -- (axis cs:0.815,0.35,0.35) -- (axis cs:0.82,0.345,0.35) -- (axis cs:0.83,0.345,0.35) -- (axis cs:0.84,0.345,0.35) -- (axis cs:0.85,0.345,0.35) -- (axis cs:0.86,0.345,0.35) -- (axis cs:0.87,0.345,0.35) -- (axis cs:0.88,0.345,0.35) -- (axis cs:0.89,0.345,0.35) -- (axis cs:0.9,0.345,0.35) -- (axis cs:0.91,0.345,0.35) -- (axis cs:0.92,0.345,0.35) -- (axis cs:0.93,0.345,0.35) -- (axis cs:0.94,0.345,0.35) -- (axis cs:0.95,0.345,0.35) -- (axis cs:0.96,0.345,0.35) -- (axis cs:0.97,0.345,0.35) -- (axis cs:0.98,0.345,0.35) -- (axis cs:0.99,0.345,0.35) -- (axis cs:1.0,0.345,0.35) -- (axis cs:1.0,0.0,0.35) -- (axis cs:0.0,0.0,0.35) -- cycle
;

      \draw[ultra thick,black] (axis cs:0.0,1.0,0.35) -- (axis cs:1.0,1.0,0.35) -- (axis cs:1.0,0.0,0.35);
   \end{scope}

   % bottom of RxO (front, negative y half only)
   \addplot3[
           surf,
           opacity = 1.0,
           samples=21,
           samples y=21,
           domain=0:1,y domain=-1:0,
           z buffer=sort]
       ({x},
        {(0.5+0.15*(y))},
        {max(0.0, ((8*x^3 - 11*x^2 + 4*x)*(1-(0.5+0.15*(y))^2) + (x)*((0.5+0.15*(y))^2))
           - (0.1 + 0.04*abs(24*x^2 -22*x + 4))*max(0.00001,(1-y^2))^0.5)});

   % top of RxO (back, positive y half)
   \addplot3[
           surf,
           opacity = 1.0,
           samples=33,
           samples y=7,
           domain=0:0.8,y domain=0:1,
           z buffer=sort]
       ({x},
        {(0.5+0.15*(y))},
        {((8*x^3 - 11*x^2 + 4*x)*(1-(0.5+0.15*(y))^2) + (x)*((0.5+0.15*(y))^2)) + 0.05*(max(0.00001,1-(2*y)^2-2*(3*(x-0.5))^2))^0.5});

   % top of RxO (front, negative y half)
   \addplot3[
           surf,
           opacity = 1.0,
           samples=41,
           samples y=7,
           domain=0:1,y domain=-1:0,
           z buffer=sort]
       ({x},
        {(0.5+0.15*(y))},
        {((8*x^3 - 11*x^2 + 4*x)*(1-(0.5+0.15*(y))^2) + (x)*((0.5+0.15*(y))^2)) + 0.05*(max(0.00001,1-(2*y)^2-2*(3*(x-0.5))^2))^0.5});
   
   % TRY ADDING G-MANIFOLD
   % top of RxO (positive y half)
   %\addplot3[
   %        surf,
   %        red,
   %        opacity=0.5,
   %        samples=41,
   %        domain=0:1,y domain=0:1,
   %        z buffer=sort]
   %    ({x},
   %     {y},
   %     {((8*x^3 - 11*x^2 + 4*x)*(1-y^2) + (x)*(y^2))});

   % draw lower box
   % front
   \draw[thick,blue,fill=blue!50,opacity=0.5]
      (axis cs:0,0,0) -- (axis cs:0,0,0.25) -- (axis cs:1,0,0.25) -- (axis cs:1,0,0.0) -- cycle;
   % side (left)
   \draw[thick,blue,fill=blue!50,opacity=0.5]
      (axis cs:0,0,0) -- (axis cs:0,0,0.25) -- (axis cs:0,1,0.25) -- (axis cs:0,1,0.0) -- cycle;

   % part of eo top surface IN FRONT of RxO
   \fill[blue!50,opacity=0.5,even odd rule]
(axis cs:0.0,0.655,0.25) -- (axis cs:0.01,0.655,0.25) -- (axis cs:0.02,0.655,0.25) -- (axis cs:0.03,0.655,0.25) -- (axis cs:0.04,0.655,0.25) -- (axis cs:0.05,0.655,0.25) -- (axis cs:0.06,0.655,0.25) -- (axis cs:0.07,0.655,0.25) -- (axis cs:0.08,0.655,0.25) -- (axis cs:0.09,0.655,0.25) -- (axis cs:0.1,0.655,0.25) -- (axis cs:0.11,0.655,0.25) -- (axis cs:0.12,0.655,0.25) -- (axis cs:0.125,0.65,0.25) -- (axis cs:0.125,0.64,0.25) -- (axis cs:0.125,0.63,0.25) -- (axis cs:0.12,0.625,0.25) -- (axis cs:0.115,0.62,0.25) -- (axis cs:0.115,0.61,0.25) -- (axis cs:0.115,0.6,0.25) -- (axis cs:0.115,0.5900000000000001,0.25) -- (axis cs:0.115,0.5800000000000001,0.25) -- (axis cs:0.11,0.575,0.25) -- (axis cs:0.105,0.5700000000000001,0.25) -- (axis cs:0.105,0.56,0.25) -- (axis cs:0.105,0.55,0.25) -- (axis cs:0.105,0.54,0.25) -- (axis cs:0.105,0.53,0.25) -- (axis cs:0.105,0.52,0.25) -- (axis cs:0.105,0.51,0.25) -- (axis cs:0.105,0.5,0.25) -- (axis cs:0.1,0.495,0.25) -- (axis cs:0.095,0.49,0.25) -- (axis cs:0.095,0.48,0.25) -- (axis cs:0.095,0.47,0.25) -- (axis cs:0.095,0.45999999999999996,0.25) -- (axis cs:0.095,0.44999999999999996,0.25) -- (axis cs:0.095,0.43999999999999995,0.25) -- (axis cs:0.095,0.43000000000000005,0.25) -- (axis cs:0.095,0.42000000000000004,0.25) -- (axis cs:0.095,0.41000000000000003,0.25) -- (axis cs:0.095,0.4,0.25) -- (axis cs:0.095,0.39,0.25) -- (axis cs:0.095,0.38,0.25) -- (axis cs:0.09,0.375,0.25) -- (axis cs:0.085,0.37,0.25) -- (axis cs:0.085,0.36,0.25) -- (axis cs:0.085,0.35,0.25) -- (axis cs:0.09,0.345,0.25) -- (axis cs:0.1,0.345,0.25) -- (axis cs:0.11,0.345,0.25) -- (axis cs:0.12,0.345,0.25) -- (axis cs:0.13,0.345,0.25) -- (axis cs:0.14,0.345,0.25) -- (axis cs:0.15,0.345,0.25) -- (axis cs:0.16,0.345,0.25) -- (axis cs:0.17,0.345,0.25) -- (axis cs:0.18,0.345,0.25) -- (axis cs:0.19,0.345,0.25) -- (axis cs:0.2,0.345,0.25) -- (axis cs:0.21,0.345,0.25) -- (axis cs:0.22,0.345,0.25) -- (axis cs:0.23,0.345,0.25) -- (axis cs:0.24,0.345,0.25) -- (axis cs:0.25,0.345,0.25) -- (axis cs:0.26,0.345,0.25) -- (axis cs:0.27,0.345,0.25) -- (axis cs:0.28,0.345,0.25) -- (axis cs:0.29,0.345,0.25) -- (axis cs:0.3,0.345,0.25) -- (axis cs:0.31,0.345,0.25) -- (axis cs:0.32,0.345,0.25) -- (axis cs:0.33,0.345,0.25) -- (axis cs:0.34,0.345,0.25) -- (axis cs:0.35,0.345,0.25) -- (axis cs:0.36,0.345,0.25) -- (axis cs:0.37,0.345,0.25) -- (axis cs:0.38,0.345,0.25) -- (axis cs:0.39,0.345,0.25) -- (axis cs:0.4,0.345,0.25) -- (axis cs:0.41,0.345,0.25) -- (axis cs:0.42,0.345,0.25) -- (axis cs:0.43,0.345,0.25) -- (axis cs:0.44,0.345,0.25) -- (axis cs:0.45,0.345,0.25) -- (axis cs:0.46,0.345,0.25) -- (axis cs:0.47,0.345,0.25) -- (axis cs:0.48,0.345,0.25) -- (axis cs:0.49,0.345,0.25) -- (axis cs:0.5,0.345,0.25) -- (axis cs:0.51,0.345,0.25) -- (axis cs:0.52,0.345,0.25) -- (axis cs:0.53,0.345,0.25) -- (axis cs:0.54,0.345,0.25) -- (axis cs:0.545,0.35,0.25) -- (axis cs:0.545,0.36,0.25) -- (axis cs:0.55,0.365,0.25) -- (axis cs:0.555,0.37,0.25) -- (axis cs:0.555,0.38,0.25) -- (axis cs:0.56,0.385,0.25) -- (axis cs:0.565,0.39,0.25) -- (axis cs:0.565,0.4,0.25) -- (axis cs:0.57,0.405,0.25) -- (axis cs:0.575,0.41000000000000003,0.25) -- (axis cs:0.58,0.41500000000000004,0.25) -- (axis cs:0.585,0.42000000000000004,0.25) -- (axis cs:0.59,0.42500000000000004,0.25) -- (axis cs:0.595,0.43000000000000005,0.25) -- (axis cs:0.6,0.43500000000000005,0.25) -- (axis cs:0.61,0.43500000000000005,0.25) -- (axis cs:0.62,0.43500000000000005,0.25) -- (axis cs:0.63,0.43500000000000005,0.25) -- (axis cs:0.64,0.43500000000000005,0.25) -- (axis cs:0.645,0.43999999999999995,0.25) -- (axis cs:0.65,0.44499999999999995,0.25) -- (axis cs:0.66,0.44499999999999995,0.25) -- (axis cs:0.67,0.44499999999999995,0.25) -- (axis cs:0.675,0.43999999999999995,0.25) -- (axis cs:0.68,0.43500000000000005,0.25) -- (axis cs:0.685,0.43000000000000005,0.25) -- (axis cs:0.69,0.42500000000000004,0.25) -- (axis cs:0.695,0.42000000000000004,0.25) -- (axis cs:0.7,0.41500000000000004,0.25) -- (axis cs:0.705,0.41000000000000003,0.25) -- (axis cs:0.71,0.405,0.25) -- (axis cs:0.715,0.4,0.25) -- (axis cs:0.72,0.395,0.25) -- (axis cs:0.725,0.39,0.25) -- (axis cs:0.725,0.38,0.25) -- (axis cs:0.73,0.375,0.25) -- (axis cs:0.735,0.37,0.25) -- (axis cs:0.735,0.36,0.25) -- (axis cs:0.74,0.355,0.25) -- (axis cs:0.745,0.35,0.25) -- (axis cs:0.75,0.345,0.25) -- (axis cs:0.76,0.345,0.25) -- (axis cs:0.77,0.345,0.25) -- (axis cs:0.78,0.345,0.25) -- (axis cs:0.79,0.345,0.25) -- (axis cs:0.8,0.345,0.25) -- (axis cs:0.81,0.345,0.25) -- (axis cs:0.82,0.345,0.25) -- (axis cs:0.83,0.345,0.25) -- (axis cs:0.84,0.345,0.25) -- (axis cs:0.85,0.345,0.25) -- (axis cs:0.86,0.345,0.25) -- (axis cs:0.87,0.345,0.25) -- (axis cs:0.88,0.345,0.25) -- (axis cs:0.89,0.345,0.25) -- (axis cs:0.9,0.345,0.25) -- (axis cs:0.91,0.345,0.25) -- (axis cs:0.92,0.345,0.25) -- (axis cs:0.93,0.345,0.25) -- (axis cs:0.94,0.345,0.25) -- (axis cs:0.95,0.345,0.25) -- (axis cs:0.96,0.345,0.25) -- (axis cs:0.97,0.345,0.25) -- (axis cs:0.98,0.345,0.25) -- (axis cs:0.99,0.345,0.25) -- (axis cs:1.0,0.345,0.25) -- (axis cs:1.0,0.0,0.25) -- (axis cs:0.0,0.0,0.25) -- cycle
;

   
   % part of pbottom manifold IN FRONT OF RxO
   \begin{scope}
      \clip (axis cs:0.0,1.0,0.35) -- (axis cs:0.8,1.0,0.35) -- (axis cs:0.8,0.5,0.35) -- (axis cs:1.0,0.5,0.35)
         -- (axis cs:1.0,0.0,0.35) -- (axis cs:0.0,0.0,0.35) -- cycle;
      \fill[red!50,opacity=0.5,even odd rule]
(axis cs:0.0,0.655,0.35) -- (axis cs:0.01,0.655,0.35) -- (axis cs:0.02,0.655,0.35) -- (axis cs:0.03,0.655,0.35) -- (axis cs:0.04,0.655,0.35) -- (axis cs:0.05,0.655,0.35) -- (axis cs:0.06,0.655,0.35) -- (axis cs:0.07,0.655,0.35) -- (axis cs:0.08,0.655,0.35) -- (axis cs:0.09,0.655,0.35) -- (axis cs:0.1,0.655,0.35) -- (axis cs:0.11,0.655,0.35) -- (axis cs:0.12,0.655,0.35) -- (axis cs:0.13,0.655,0.35) -- (axis cs:0.14,0.655,0.35) -- (axis cs:0.15,0.655,0.35) -- (axis cs:0.16,0.655,0.35) -- (axis cs:0.17,0.655,0.35) -- (axis cs:0.18,0.655,0.35) -- (axis cs:0.19,0.655,0.35) -- (axis cs:0.2,0.655,0.35) -- (axis cs:0.21,0.655,0.35) -- (axis cs:0.22,0.655,0.35) -- (axis cs:0.23,0.655,0.35) -- (axis cs:0.235,0.65,0.35) -- (axis cs:0.23,0.645,0.35) -- (axis cs:0.225,0.64,0.35) -- (axis cs:0.225,0.63,0.35) -- (axis cs:0.22,0.625,0.35) -- (axis cs:0.215,0.62,0.35) -- (axis cs:0.215,0.61,0.35) -- (axis cs:0.21,0.605,0.35) -- (axis cs:0.205,0.6,0.35) -- (axis cs:0.205,0.5900000000000001,0.35) -- (axis cs:0.205,0.5800000000000001,0.35) -- (axis cs:0.2,0.575,0.35) -- (axis cs:0.195,0.5700000000000001,0.35) -- (axis cs:0.195,0.56,0.35) -- (axis cs:0.195,0.55,0.35) -- (axis cs:0.19,0.5449999999999999,0.35) -- (axis cs:0.185,0.54,0.35) -- (axis cs:0.185,0.53,0.35) -- (axis cs:0.185,0.52,0.35) -- (axis cs:0.18,0.515,0.35) -- (axis cs:0.175,0.51,0.35) -- (axis cs:0.175,0.5,0.35) -- (axis cs:0.175,0.49,0.35) -- (axis cs:0.175,0.48,0.35) -- (axis cs:0.17,0.475,0.35) -- (axis cs:0.165,0.47,0.35) -- (axis cs:0.165,0.45999999999999996,0.35) -- (axis cs:0.165,0.44999999999999996,0.35) -- (axis cs:0.165,0.43999999999999995,0.35) -- (axis cs:0.165,0.43000000000000005,0.35) -- (axis cs:0.16,0.42500000000000004,0.35) -- (axis cs:0.155,0.42000000000000004,0.35) -- (axis cs:0.155,0.41000000000000003,0.35) -- (axis cs:0.155,0.4,0.35) -- (axis cs:0.155,0.39,0.35) -- (axis cs:0.155,0.38,0.35) -- (axis cs:0.155,0.37,0.35) -- (axis cs:0.155,0.36,0.35) -- (axis cs:0.15,0.355,0.35) -- (axis cs:0.145,0.35,0.35) -- (axis cs:0.15,0.345,0.35) -- (axis cs:0.16,0.345,0.35) -- (axis cs:0.17,0.345,0.35) -- (axis cs:0.18,0.345,0.35) -- (axis cs:0.19,0.345,0.35) -- (axis cs:0.2,0.345,0.35) -- (axis cs:0.21,0.345,0.35) -- (axis cs:0.22,0.345,0.35) -- (axis cs:0.23,0.345,0.35) -- (axis cs:0.24,0.345,0.35) -- (axis cs:0.25,0.345,0.35) -- (axis cs:0.26,0.345,0.35) -- (axis cs:0.27,0.345,0.35) -- (axis cs:0.28,0.345,0.35) -- (axis cs:0.29,0.345,0.35) -- (axis cs:0.3,0.345,0.35) -- (axis cs:0.31,0.345,0.35) -- (axis cs:0.32,0.345,0.35) -- (axis cs:0.33,0.345,0.35) -- (axis cs:0.34,0.345,0.35) -- (axis cs:0.35,0.345,0.35) -- (axis cs:0.36,0.345,0.35) -- (axis cs:0.37,0.345,0.35) -- (axis cs:0.38,0.345,0.35) -- (axis cs:0.39,0.345,0.35) -- (axis cs:0.4,0.345,0.35) -- (axis cs:0.41,0.345,0.35) -- (axis cs:0.415,0.35,0.35) -- (axis cs:0.415,0.36,0.35) -- (axis cs:0.415,0.37,0.35) -- (axis cs:0.415,0.38,0.35) -- (axis cs:0.415,0.39,0.35) -- (axis cs:0.415,0.4,0.35) -- (axis cs:0.415,0.41000000000000003,0.35) -- (axis cs:0.415,0.42000000000000004,0.35) -- (axis cs:0.42,0.42500000000000004,0.35) -- (axis cs:0.43,0.42500000000000004,0.35) -- (axis cs:0.435,0.43000000000000005,0.35) -- (axis cs:0.44,0.43500000000000005,0.35) -- (axis cs:0.45,0.43500000000000005,0.35) -- (axis cs:0.46,0.43500000000000005,0.35) -- (axis cs:0.465,0.43999999999999995,0.35) -- (axis cs:0.47,0.44499999999999995,0.35) -- (axis cs:0.475,0.44999999999999996,0.35) -- (axis cs:0.48,0.45499999999999996,0.35) -- (axis cs:0.49,0.45499999999999996,0.35) -- (axis cs:0.495,0.45999999999999996,0.35) -- (axis cs:0.5,0.46499999999999997,0.35) -- (axis cs:0.505,0.47,0.35) -- (axis cs:0.51,0.475,0.35) -- (axis cs:0.515,0.48,0.35) -- (axis cs:0.515,0.49,0.35) -- (axis cs:0.52,0.495,0.35) -- (axis cs:0.525,0.5,0.35) -- (axis cs:0.53,0.505,0.35) -- (axis cs:0.535,0.51,0.35) -- (axis cs:0.535,0.52,0.35) -- (axis cs:0.535,0.53,0.35) -- (axis cs:0.535,0.54,0.35) -- (axis cs:0.535,0.55,0.35) -- (axis cs:0.53,0.5549999999999999,0.35) -- (axis cs:0.525,0.56,0.35) -- (axis cs:0.52,0.565,0.35) -- (axis cs:0.51,0.565,0.35) -- (axis cs:0.5,0.565,0.35) -- (axis cs:0.495,0.5700000000000001,0.35) -- (axis cs:0.49,0.575,0.35) -- (axis cs:0.48,0.575,0.35) -- (axis cs:0.47,0.575,0.35) -- (axis cs:0.46,0.575,0.35) -- (axis cs:0.455,0.5800000000000001,0.35) -- (axis cs:0.455,0.5900000000000001,0.35) -- (axis cs:0.46,0.595,0.35) -- (axis cs:0.465,0.6,0.35) -- (axis cs:0.47,0.605,0.35) -- (axis cs:0.475,0.61,0.35) -- (axis cs:0.48,0.615,0.35) -- (axis cs:0.485,0.62,0.35) -- (axis cs:0.49,0.625,0.35) -- (axis cs:0.495,0.63,0.35) -- (axis cs:0.5,0.635,0.35) -- (axis cs:0.51,0.635,0.35) -- (axis cs:0.515,0.64,0.35) -- (axis cs:0.52,0.645,0.35) -- (axis cs:0.53,0.645,0.35) -- (axis cs:0.54,0.645,0.35) -- (axis cs:0.545,0.65,0.35) -- (axis cs:0.55,0.655,0.35) -- (axis cs:0.56,0.655,0.35) -- (axis cs:0.57,0.655,0.35) -- (axis cs:0.58,0.655,0.35) -- (axis cs:0.59,0.655,0.35) -- (axis cs:0.595,0.65,0.35) -- (axis cs:0.6,0.645,0.35) -- (axis cs:0.61,0.645,0.35) -- (axis cs:0.62,0.645,0.35) -- (axis cs:0.63,0.645,0.35) -- (axis cs:0.635,0.64,0.35) -- (axis cs:0.64,0.635,0.35) -- (axis cs:0.65,0.635,0.35) -- (axis cs:0.655,0.63,0.35) -- (axis cs:0.66,0.625,0.35) -- (axis cs:0.67,0.625,0.35) -- (axis cs:0.675,0.62,0.35) -- (axis cs:0.68,0.615,0.35) -- (axis cs:0.685,0.61,0.35) -- (axis cs:0.69,0.605,0.35) -- (axis cs:0.695,0.6,0.35) -- (axis cs:0.7,0.595,0.35) -- (axis cs:0.705,0.5900000000000001,0.35) -- (axis cs:0.71,0.585,0.35) -- (axis cs:0.715,0.5800000000000001,0.35) -- (axis cs:0.72,0.575,0.35) -- (axis cs:0.725,0.5700000000000001,0.35) -- (axis cs:0.73,0.565,0.35) -- (axis cs:0.735,0.56,0.35) -- (axis cs:0.74,0.5549999999999999,0.35) -- (axis cs:0.745,0.55,0.35) -- (axis cs:0.745,0.54,0.35) -- (axis cs:0.75,0.5349999999999999,0.35) -- (axis cs:0.755,0.53,0.35) -- (axis cs:0.755,0.52,0.35) -- (axis cs:0.76,0.515,0.35) -- (axis cs:0.765,0.51,0.35) -- (axis cs:0.765,0.5,0.35) -- (axis cs:0.77,0.495,0.35) -- (axis cs:0.775,0.49,0.35) -- (axis cs:0.775,0.48,0.35) -- (axis cs:0.78,0.475,0.35) -- (axis cs:0.785,0.47,0.35) -- (axis cs:0.785,0.45999999999999996,0.35) -- (axis cs:0.785,0.44999999999999996,0.35) -- (axis cs:0.79,0.44499999999999995,0.35) -- (axis cs:0.795,0.43999999999999995,0.35) -- (axis cs:0.795,0.43000000000000005,0.35) -- (axis cs:0.795,0.42000000000000004,0.35) -- (axis cs:0.8,0.41500000000000004,0.35) -- (axis cs:0.805,0.41000000000000003,0.35) -- (axis cs:0.805,0.4,0.35) -- (axis cs:0.805,0.39,0.35) -- (axis cs:0.805,0.38,0.35) -- (axis cs:0.81,0.375,0.35) -- (axis cs:0.815,0.37,0.35) -- (axis cs:0.815,0.36,0.35) -- (axis cs:0.815,0.35,0.35) -- (axis cs:0.82,0.345,0.35) -- (axis cs:0.83,0.345,0.35) -- (axis cs:0.84,0.345,0.35) -- (axis cs:0.85,0.345,0.35) -- (axis cs:0.86,0.345,0.35) -- (axis cs:0.87,0.345,0.35) -- (axis cs:0.88,0.345,0.35) -- (axis cs:0.89,0.345,0.35) -- (axis cs:0.9,0.345,0.35) -- (axis cs:0.91,0.345,0.35) -- (axis cs:0.92,0.345,0.35) -- (axis cs:0.93,0.345,0.35) -- (axis cs:0.94,0.345,0.35) -- (axis cs:0.95,0.345,0.35) -- (axis cs:0.96,0.345,0.35) -- (axis cs:0.97,0.345,0.35) -- (axis cs:0.98,0.345,0.35) -- (axis cs:0.99,0.345,0.35) -- (axis cs:1.0,0.345,0.35) -- (axis cs:1.0,0.0,0.35) -- (axis cs:0.0,0.0,0.35) -- cycle
;

      \draw[ultra thick,black] (axis cs:1.0,0.35,0.35) -- (axis cs:1.0,0.0,0.35); % ugh
   \end{scope}

   % add green box corner on top
   %\fill[green!50]
(axis cs:0.8,0.5,1.0) -- (axis cs:1.0,0.5,1.0) -- (axis cs:0.99,0.5,0.9534689999999999) -- (axis cs:0.98,0.5,0.9088520000000001) -- (axis cs:0.97,0.5,0.8661129999999997) -- (axis cs:0.96,0.5,0.8252159999999997) -- (axis cs:0.9500000000000001,0.5,0.7861250000000014) -- (axis cs:0.9400000000000001,0.5,0.7488039999999996) -- (axis cs:0.93,0.5,0.7132169999999998) -- (axis cs:0.92,0.5,0.6793279999999994) -- (axis cs:0.91,0.5,0.6471009999999996) -- (axis cs:0.9,0.5,0.6165000000000005) -- (axis cs:0.89,0.5,0.5874889999999999) -- (axis cs:0.88,0.5,0.560032) -- (axis cs:0.87,0.5,0.5340929999999992) -- (axis cs:0.86,0.5,0.5096360000000009) -- (axis cs:0.85,0.5,0.4866250000000003) -- (axis cs:0.84,0.5,0.4650240000000002) -- (axis cs:0.8300000000000001,0.5,0.4447970000000008) -- (axis cs:0.8200000000000001,0.5,0.42590799999999945) -- (axis cs:0.81,0.5,0.4083209999999996) -- (axis cs:0.8,0.5,0.3920000000000002) -- cycle;
\fill[green!50]
(axis cs:0.8,1,1.0) -- (axis cs:0.8,0.5,1.0) -- (axis cs:0.8,0.5,0.3920000000000002) -- (axis cs:0.8,0.51,0.3974944000000002) -- (axis cs:0.8,0.52,0.4030976000000002) -- (axis cs:0.8,0.53,0.4088096000000002) -- (axis cs:0.8,0.54,0.4146304000000002) -- (axis cs:0.8,0.55,0.4205600000000002) -- (axis cs:0.8,0.56,0.42659840000000016) -- (axis cs:0.8,0.5700000000000001,0.43274560000000023) -- (axis cs:0.8,0.58,0.4390016000000001) -- (axis cs:0.8,0.59,0.44536640000000016) -- (axis cs:0.8,0.6,0.45184000000000013) -- (axis cs:0.8,0.61,0.4584224000000001) -- (axis cs:0.8,0.62,0.4651136000000001) -- (axis cs:0.8,0.63,0.47191360000000016) -- (axis cs:0.8,0.64,0.4788224000000002) -- (axis cs:0.8,0.65,0.48584000000000016) -- (axis cs:0.8,0.66,0.4929664000000002) -- (axis cs:0.8,0.67,0.5002016000000002) -- (axis cs:0.8,0.68,0.5075456000000002) -- (axis cs:0.8,0.6900000000000001,0.5149984000000002) -- (axis cs:0.8,0.7000000000000001,0.5225600000000002) -- (axis cs:0.8,0.71,0.5302304000000001) -- (axis cs:0.8,0.72,0.5380096000000001) -- (axis cs:0.8,0.73,0.5458976000000001) -- (axis cs:0.8,0.74,0.5538944000000001) -- (axis cs:0.8,0.75,0.562) -- (axis cs:0.8,0.76,0.5702144000000001) -- (axis cs:0.8,0.77,0.5785376000000001) -- (axis cs:0.8,0.78,0.5869696000000001) -- (axis cs:0.8,0.79,0.5955104000000002) -- (axis cs:0.8,0.8,0.6041600000000001) -- (axis cs:0.8,0.81,0.6129184000000002) -- (axis cs:0.8,0.8200000000000001,0.6217856000000002) -- (axis cs:0.8,0.8300000000000001,0.6307616000000001) -- (axis cs:0.8,0.84,0.6398464) -- (axis cs:0.8,0.85,0.6490400000000001) -- (axis cs:0.8,0.86,0.6583424000000001) -- (axis cs:0.8,0.87,0.6677536000000001) -- (axis cs:0.8,0.88,0.6772736000000001) -- (axis cs:0.8,0.89,0.6869024) -- (axis cs:0.8,0.9,0.6966400000000001) -- (axis cs:0.8,0.91,0.7064864000000001) -- (axis cs:0.8,0.92,0.7164416000000001) -- (axis cs:0.8,0.93,0.7265056000000001) -- (axis cs:0.8,0.9400000000000001,0.7366784000000002) -- (axis cs:0.8,0.9500000000000001,0.7469600000000001) -- (axis cs:0.8,0.96,0.7573504000000001) -- (axis cs:0.8,0.97,0.7678496000000001) -- (axis cs:0.8,0.98,0.7784576000000001) -- (axis cs:0.8,0.99,0.7891744) -- (axis cs:0.8,1.0,0.8) -- cycle;
\fill[green!50]
(axis cs:0.8,1.0,1) -- (axis cs:1.0,1.0,1) -- (axis cs:1.0,0.5,1) -- (axis cs:0.8,0.5,1) -- cycle;

   
   % add ro bump on top!
   % xc,yc = 0.5,0.5
   % xrad = 1.0/(3*(2**0.5))
   % yrad = 0.075
   %\addplot3[
   %        surf,
   %        opacity=1.0,
   %        samples=11, samples y=21,
   %        domain=0.0:1, y domain=0:360,
   %        z buffer=sort]
   %(
   %   {0.5+0.235702*x*cos(y)},
   %   {0.5+0.075*x*sin(y)},
   %   {((8*(0.5+0.235702*x*cos(y))^3
   %      - 11*(0.5+0.235702*x*cos(y))^2
   %      + 4*(0.5+0.235702*x*cos(y)))*(1-(0.5+0.075*x*sin(y))^2)
   %      + (0.5+0.235702*x*cos(y))*((0.5+0.075*x*sin(y))^2))
   %      + 0.05*max(0.0000001,(1-x^2))^0.5)}
   %);

   % g boundary
   %\addplot3[thick] table[x=x, y=y, z=z] {figs/family-composite/g-boundary.txt};
   %\addplot3[thick] table[x=x, y=y, z=z] {figs/family-composite/g-oe-intersection-noboundary.txt};
   %\addplot3[thick] table[x=x, y=y, z=z] {figs/family-composite/g-ro-intersection.txt};
   %\addplot3[thick] table[x=x, y=y, z=z] {figs/family-composite/rre-gmanifold-boundary.txt};
   %\addplot3[thick,smooth] table[x=x, y=y, z=z] {figs/family-composite/path-transfer.txt};



   %\addplot3[thick,blue] table[x=x, y=y, z=z] {figs/family-composite/pbottom-ro-intersection.txt};
   
   \addplot3[thick,black] table[x=x, y=y, z=z] {figs/family-composite/pbottom-ro-intersection.txt};
   % manually draw green box intersection boundary
   \addplot3[thick,black] coordinates { (0.8,1.0,0.35) (0.8,0.5,0.35) (1.0,0.5,0.35) };
   % front manifold boundary
   \addplot3[thick,black] coordinates { (0.0,1.0,0.35) (0.0,0.0,0.35) (1.0,0.0,0.35) };
   
   %\addplot3[thick,red,fill=red!50] table[x=x, y=y, z=z] {figs/family-composite/oe-ro-intersection.txt};
   %\addplot3[thick,fill=blue,opacity=0.5] table[x=x, y=y, z=z] {figs/family-composite/g-oe-intersection.txt};

   % draw path
   \addplot3[thick,smooth] table[x=x, y=y, z=z] {figs/family-composite/path-transit2.txt};
   \node[inner sep=1pt,fill=black,circle] at (axis cs:0.42,0.37,0.35) {}; % release
   \node[inner sep=1pt,fill=black,circle] at (axis cs:0.90,0.10,0.35) {}; % end

\end{axis}


\end{tikzpicture}
\end{document}
