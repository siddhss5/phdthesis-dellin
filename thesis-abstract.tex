
\begin{fullwidth}
\begin{adjustwidth}{1in}{1in}

\vphantom{.}

\vspace{0.2in}

{\LARGE \emph{Abstract}}

\vspace{0.2in}

An effective autonomous robot performing dangerous or menial tasks
will need to act under significant time and energy constraints.
At task time,
the amount of effort a robot spends planning its motion directly
detracts from its total performance.
Manipulation tasks, however, present challenges
to efficient motion planning.
Tightly coupled steps (e.g. choices of object grasps or placements)
allow poor early decisions to render subsequent steps difficult,
which this encourages longer planning horizons.
However,
an articulated robot
situated within a geometrically complex and dynamic environment
induces a high-dimensional configuration space
in which it is expensive to test for valid paths.
And since multi-step plans
require paths in changing valid subsets of configuration space,
it is difficult to reuse computation across steps.

\vspace{0.2cm}

This thesis proposes an approach to motion planning
well-suited to articulated robots
performing recurring multi-step manipulation tasks
in complex, semi-structured environments.
The high cost of edge validation in roadmap methods
motivates us to study a lazy approach to pathfinding on graphs
which decouples constructing and searching the graph
from validating its edges.
This decoupling raises two immediate questions which we address:
(a) how to allocate precious validation computation
among the unevaluated edges on the graph,
and (b) how to efficiently solve the resulting dynamic pathfinding
problem which arises as edges are validated.
We next consider the inherent tradeoff
between planning and execution cost,
and show that an objective based on utility functions
is able to effectively balance these competing goals
during lazy pathfinding.
Lastly,
we define the family motion planning problem
which captures the structure of multi-step manipulation tasks,
and propose a related utility function which allows our
motion planner to quickly find efficient solutions for such tasks.

\vspace{0.2cm}

We assemble our algorithms into an integrated manipulation planning
system,
and demonstrate its effectiveness on motion and manipulation tasks
on several robotic platforms.
We also provide open-source implementations of our algorithms
for contemporary motion planning frameworks.
While the motivation for this thesis originally derived
from manipulation,
our pathfinding algorithms are broadly applicable to problem domains
in which edge validation is expensive.
Furthermore,
the underlying similarity between lazy and dynamic settings
also renders our incremental algorithms applicable
to conventional dynamic problems such as traffic routing.

\end{adjustwidth}
\end{fullwidth}
